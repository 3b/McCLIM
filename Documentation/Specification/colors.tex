% -*- Mode: LaTeX; Package: CLIM-USER -*-

\chapter {Drawing in Color}
\label {color}

This chapter describes the \cl{:ink} drawing option and the simpler values that
can be supplied for that option, such as colors.  More complex values that have
a regular or irregular pattern in the ink are described in Chapter~\ref{designs}.

\Issue {SWM} {We need to add a thing called a ``palette'', which is simply an
abstract color map.  Palettes are primarily used as a resource for the limited
number of colors on most hosts.  Do we need to be able to used them to more
directly control color maps, to do ``color map animation'', for example?}

\issue {SWM} {We need to add a thing called a ``raster ink'', which includes
things like plane masks, pixel values, etc.  Be clear that this is platform and
device dependent.}


%% Use \tt instead of \cl, such the hairy \cl macro will blow chow
\section {The {\tt :ink} Drawing Option}

The \cl{:ink} drawing option, used with the drawing functions described in
Chapter~\ref{graphics}, can take as its value:
\begin{itemize}
\item a color,

\item an opacity or the constant \cl{+transparent-ink+},

\item the constant \cl{+foreground-ink+},

\item the constant \cl{+background-ink+},

\item a flipping ink, or

\item other values described in Chapter~\ref{designs}
\end{itemize}

More exactly, an ink can be any member of the class \cl{design}.  For now you
may think of a design as a possibly translucent color.  More general designs are
described in Chapter~\ref{designs}.

The drawing functions work by selecting a region of the drawing plane and
painting it with color.  The region to be painted is the intersection of the
shape specified by the drawing function and the \cl{:clipping-region} drawing
option, which is then transformed by the \cl{:transformation} drawing option.
The \cl{:ink} drawing option is a design that specifies a new arrangement of
colors (and opacities) in this region of the medium's drawing plane.  Any
viewports or dataports attached to this drawing plane are updated accordingly.
The \cl{:ink} drawing option is never affected by the \cl{:transformation}
drawing option nor by the medium's transformation; this ensures that stipple
patterns on adjacent sheets join seamlessly.

\issue {DCPL} {The description of how the clipping region and transformations
contribute isn't good enough.  It is true if there are no other transformations
and clipping regions present, and both are specified in the current drawing
operation.  But it doesn't say what happens if things are nested.  I'm not sure
it needs to.  Rather, I think it should just say that the the region is clipped
by the current clipping region in effect, then transformed by the current
transform in effect, and that the rules for these are discussed in the drawing
options section.}

Drawing consists conceptually of the following sequence of operations, performed
in parallel at every point in the drawing plane.  Of course, the actual
implementation does not involve an infinite (or large parallel) computation.

\begin{enumerate}
\item The design specifies a color and an opacity at the point.  These can
depend on the drawing plane's current color and opacity, on the medium's
foreground color, and on the medium's background color.

\item The color blending function is applied to the design's color and opacity
and the drawing plane's color and opacity, returning a new color and opacity for
the point.

\item The drawing plane's color and opacity at that point are set to the new
color and opacity.
\end{enumerate}

\section {Basic Designs}

\Defprotoclass {design}

A design is an object that represents a way of arranging colors and opacities in
the drawing plane.  The \cl{design} class is the protocol class for designs.
\IfYouWantClass {a} {design} {design}

The fundamental operation of the CLIM graphic drawing model
is to draw a design onto a drawing plane, thus drawing is always controlled by
designs.  The designs discussed in this chapter do the same thing at each point
in the drawing plane.  Chapter~\ref{designs} discusses more general designs and
reveals that regions are also designs.

A design can be characterized in several different ways:

All designs are either \concept{bounded} or \concept{unbounded}.  Bounded
designs are transparent everywhere beyond a certain distance from a certain
point.  Drawing a bounded design has no effect on the drawing plane outside that
distance.  Unbounded designs have points of non-zero opacity arbitrarily far
from the origin.  Drawing an unbounded design affects the entire drawing plane.

All designs are either \concept{uniform} or \concept{non-uniform}.  Uniform
designs have the same color and opacity at every point in the drawing plane.
Uniform designs are always unbounded, unless they are completely transparent.

All designs are either \concept{solid} or \concept{translucent}.  At each point
a solid design is either completely opaque or completely transparent.  A solid
design can be opaque at some points and transparent at others.  In translucent
designs, at least one point has an opacity that is intermediate between
completely opaque and transparent.

All designs are either \concept{colorless} or \concept{colored}.  Drawing a
colorless design uses a default color specified by the medium's foreground
design.  This is done by drawing with \cl{(compose-in} \cl{+foreground-ink+}
the-colorless-design\cl{)}.  See Chapter~\ref{designs} for the details of
\cl{compose-in}.

\Defpredicate {designp} {object}

Returns \term{true} if \arg{object} is a \term{design}, otherwise returns
\term{false}.


\section {Color}

\Defprotoclass {color}

A member of the class \cl{color} is a completely opaque design that represents
the intuitive definition of color: white, black, red, pale yellow, and so forth.
The visual appearance of a single point is completely described by its color.
Drawing a color sets the color of every point in the drawing plane to that
color, and sets the opacity to 1.  The \cl{color} class is the protocol class
for a color, and is a subclass of \cl{design}.
\IfYouWantClass {a} {color} {color}

All of the standard instantiable color classes provided by CLIM are immutable.

A color can be specified by three real numbers between 0 and 1 (inclusive),
giving the amounts of red, green, and blue.  Three 0's mean black; three 1's
mean white.  The intensity-hue-saturation color model is also supported, but the
red-green-blue color model is the primary model we will use in the
specification.

\Defpredicate {colorp} {object}

Returns \term{true} if \arg{object} is a \term{color}, otherwise returns
\term{false}.


The following functions create colors.  These functions produce objects that
have equivalent effects; the only difference is in how the color components are
specified.  The resulting objects are indistinguishable when drawn.  Whether
these functions use the specified values exactly or approximate them because of
limited color resolution is unspecified.  Whether these functions create a new
object or return an existing object with equivalent color component values is
unspecified.


\Defun {make-rgb-color} {red green blue}

Returns a member of class \cl{color}.  The \arg{red}, \arg{green}, and
\arg{blue} arguments are real numbers between 0 and 1 (inclusive) that specify
the values of the corresponding color components.


\Defun {make-ihs-color} {intensity hue saturation}

Returns a member of class \cl{color}.  The \arg{intensity} argument is a real
number between 0 and $\sqrt{3}$ (inclusive).  The \arg{hue} and \arg{saturation}
arguments are real numbers between 0 and 1 (inclusive).


\Defun {make-gray-color} {luminance}

Returns a member of class \cl{color}.  \arg{luminance} is a real number between
0 and 1 (inclusive).  On a black-on-white display device, 0 means black, 1 means
white, and the values in between are shades of gray.  On a white-on-black
display device, 0 means white, 1 means black, and the values in between are
shades of gray.

\issue {PM} {Why does the caller need to know the type of display?}


The following two functions comprise the color protocol.  Both of them return
the components of a color.  All subclasses of \cl{color} must implement methods
for these generic functions.

\Defgeneric {color-rgb} {color}

Returns three values, the \arg{red}, \arg{green}, and \arg{blue} components of
the \term{color} \arg{color}. The values are real numbers between 0 and 1
(inclusive).

\Defgeneric {color-ihs} {color}

Returns three values, the \arg{intensity}, \arg{hue}, and \arg{saturation}
components of the \term{color} \arg{color}.  The first value is a real number
between 0 andn $\sqrt{3}$ (inclusive).  The second and third values are real
numbers between 0 and 1 (inclusive).


\subsection {Standard Color Names and Constants}

Table~\ref{color-names} lists the commonly provided color names that can be
looked up with \cl{find-named-color}.  Application programs can define other
colors; these are provided because they are commonly used in the X Windows
community, not because there is anything special about these particular colors.
This table is a subset of the color listed in the file
\cl{/X11/R4/mit/rgb/rgb.txt}, from the X11 R4 distribution.

\begin{table}
\hrule
\vspace{1pc}
\cl{
\begin{tabular}{lllll}
alice-blue & antique-white & aquamarine \\
azure & beige & bisque \\
black & blanched-almond & blue \\
blue-violet & brown & burlywood \\
cadet-blue & chartreuse & chocolate \\
coral & cornflower-blue & cornsilk \\
cyan & dark-goldenrod & dark-green \\
dark-khaki & dark-olive-green & dark-orange \\
dark-orchid & dark-salmon & dark-sea-green \\
dark-slate-blue & dark-slate-gray & dark-turquoise \\
dark-violet & deep-pink & deep-sky-blue \\
dim-gray & dodger-blue & firebrick \\
floral-white & forest-green & gainsboro \\
ghost-white & gold & goldenrod \\
gray & green & green-yellow \\
honeydew & hot-pink & indian-red \\
ivory & khaki & lavender \\
lavender-blush & lawn-green & lemon-chiffon \\
light-blue & light-coral & light-cyan \\
light-goldenrod & light-goldenrod-yellow & light-gray \\
light-pink & light-salmon & light-sea-green \\
light-sky-blue & light-slate-blue & light-slate-gray \\
light-steel-blue & light-yellow & lime-green \\
linen & magenta & maroon \\
medium-aquamarine & medium-blue & medium-orchid \\
medium-purple & medium-sea-green & medium-slate-blue \\
medium-spring-green & medium-turquoise & medium-violet-red \\
midnight-blue & mint-cream & misty-rose \\
moccasin & navajo-white & navy-blue \\
old-lace & olive-drab & orange \\
orange-red & orchid & pale-goldenrod \\
pale-green & pale-turquoise & pale-violet-red \\
papaya-whip & peach-puff & peru \\
pink & plum & powder-blue \\
purple & red & rosy-brown \\
royal-blue & saddle-brown & salmon \\
sandy-brown & sea-green & seashell \\
sienna & sky-blue & slate-blue \\
slate-gray & snow & spring-green \\
steel-blue & tan & thistle \\
tomato & turquoise & violet \\
violet-red & wheat & white \\
white-smoke & yellow & yellow-green \\
\end{tabular}}
\caption{\label{color-names} Standard color names.}
\vspace{1pc}
\hrule
\end{table}

In addition, the following color constants are provided.

\defconst {+red+}
\defconst {+green+}
\defconst {+blue+}
\defconst {+cyan+}
\defconst {+magenta+}
\defconst {+yellow+}
\defconst {+black+}
\Defconst {+white+}

Constants corresponding to the usual definitions of red, green, blue, cyan,
magenta, yellow, black, and white.


\subsection {Contrasting Colors}

\Defun {make-contrasting-inks} {n \optional k}

If \arg{k} is not supplied, this returns a vector of \arg{n} designs with
recognizably different appearance.  Elements of the vector are guaranteed to be
acceptable values for the \cl{:ink} argument to the drawing functions, and will
not include \cl{+foreground-ink+}, \cl{+background-ink+}, or \cl{nil}.  Their
class is otherwise unspecified.  The vector is a fresh object that may be
modified.

If \arg{k} is supplied, it must be an integer between 0 and $\arg{n}-1$
(inclusive), in which case \cl{make-contrasting-inks} returns the \arg{k}'th
design rather than returning a vector of designs.

If the implementation does not have \arg{n} different contrasting inks,
\cl{make-contrasting-inks} signals an error.  This will not happen unless
\arg{n} is greater than eight.

The rendering of the design may be a color or a stippled pattern, depending on
whether the output medium supports color.

\Defgeneric {contrasting-inks-limit} {port}

Returns the number of contrasting colors (or stipple patterns if \arg{port} is
monochrome or grayscale) that can be rendered on any medium on the \term{port}
\arg{port}.  Implementations are encouraged to make this as large as possible,
but it must be at least 8.  All classes that obey the medium protocol must
implement a method for this generic function.


\section {Opacity}

\Defprotoclass {opacity}

A member of the class \cl{opacity} is a completely colorless design that is
typically used as the second argument to \cl{compose-in} to adjust the opacity
of another design.  See Chapter~\ref{designs} for the details of
\cl{compose-in}.  The \cl{opacity} class is the protocol class for an opacity,
and is a subclass of \cl{design}.
\IfYouWantClass {an} {opacity} {opacity}

All of the standard instantiable opacity classes provided by CLIM are immutable.

Opacity controls how graphical output covers previous output.  Opacity can vary
from totally opaque to totally transparent.  Intermediate opacity values result
in color blending so that the earlier picture shows through what is drawn on top
of it.

An opacity may be specified by a real number between 0 and 1 (inclusive).  0 is
completely transparent, 1 is completely opaque, fractions are translucent.  The
opacity of a design is the degree to which it hides the previous contents of the
drawing plane when it is drawn.

The fully transparent and fully opaque opacity levels (that is, opacities 0 and
1) must always be supported, but a valid CLIM implementation might only support
a handful of opacity levels in between (including none).  A valid CLIM
implementation might implement color blending and unsaturated colors by
stippling, although it is preferred, when possible, for a viewport to display a
uniform color as a uniform color rather than as a perceptible stipple.

\Defpredicate {opacityp} {object}

Returns \term{true} if \arg{object} is an \term{opacity}, otherwise returns
\term{false}.


The following function returns an opacity:

\Defun {make-opacity} {value}

Returns a member of class \cl{opacity} whose opacity is \arg{value}, which is a
real number in the range from 0 to 1 (inclusive), where 0 is fully transparent
and 1 is fully opaque.

\Defconst {+transparent-ink+}

An fully transparent ink, that is, an opacity whose value is 0.  This is
typically used as the ``background'' ink in a call to \cl{make-pattern}.


The following function returns the sole component of an opacity.  This is the
only function in the opacity protocol.  All subclasses of \cl{opacity} must
implement methods for this generic function.

\Defgeneric {opacity-value} {opacity}

Returns the opacity value of the \term{opacity} \arg{opacity}, which is a real
number in the range from 0 to 1 (inclusive).


\section {Color Blending}

Drawing a design that is not completely opaque at all points allows the previous
contents of the drawing plane to show through.  The simplest case is drawing a
solid design:  where the design is opaque, it replaces the previous contents of
the drawing plane; where the design is transparent, it leaves the drawing plane
unchanged.  In the more general case of drawing a translucent design, the
resulting color is a blend of the design's color and the previous color of the
drawing plane.  For purposes of color blending, the drawn design is called the
foreground and the drawing plane is called the background.

The function \cl{compose-over} performs a similar operation: it combines two
designs to produce a design, rather than combining a design and the contents of
the drawing plane to produce the new contents of the drawing plane.  For
purposes of color blending, the first argument to \cl{compose-over} is called
the foreground and the second argument is called the background.  See
Chapter~\ref{designs} for the details of \cl{compose-over}.

Color blending is defined by an ideal function
${\cal F}\colon (r_1,g_1,b_1,o_1,r_2,g_2,b_2,o_2)\rightarrow (r_3,g_3,b_3,o_3)$
that operates on the color and opacity at a single point.
$(r_1,g_1,b_1,o_1)$ are the foreground color and opacity.
$(r_2,g_2,b_2,o_2)$ are the background color and opacity.
$(r_3,g_3,b_3,o_3)$ are the resulting color and opacity.  The color
blending function ${\cal F}$ is conceptually applied at every point in the
drawing plane.

${\cal F}$ performs linear interpolation on all four components:
\[ \begin{array}{rcrcl}
o_3 & = &        o_1 & + & (1 - o_1) * o_2 \\
r_3 & = & (r_1 * o_1 & + & (1 - o_1) * o_2 * r_2) / o_3 \\
g_3 & = & (g_1 * o_1 & + & (1 - o_1) * o_2 * g_2) / o_3 \\
b_3 & = & (b_1 * o_1 & + & (1 - o_1) * o_2 * b_2) / o_3
\end{array} \]
Note that if $o_3$ is zero, these equations would divide zero by zero.
In that case $r_3$, $g_3$, and $b_3$ are defined to be zero.

CLIM requires that ${\cal F}$ be implemented exactly if $o_1$ is zero or one or
if $o_2$ is zero.  If $o_1$ is zero, the result is the background.  If $o_1$ is
one or $o_2$ is zero, the result is the foreground.  For fractional opacity
values, an implementation can deviate from the ideal color blending function
either because the implementation has limited opacity resolution or because the
implementation can compute a different color blending function much more
quickly.

If a medium's background design is not completely opaque at all points, the
consequences are unspecified.  Consequently, a drawing plane is always opaque
and drawing can use simplified color blending that assumes $o_2 = 1$ and
$o_3 = 1$.  However, \cl{compose-over} must handle a non-opaque background
correctly. 

Note that these $(r,g,b,o)$ quadruples of real numbers between 0 and 1 are
mathematical and an implementation need not store information in this form.
Most implementations are expected to use a different representation.


\section {Indirect Inks}

Drawing with an \concept{indirect ink} is the same as drawing another design
named directly.  For example, \cl{+foreground-ink+} is a design that draws the
medium's foreground design and is the default value of the \cl{:ink} drawing
option.  Indirect inks exist for the benefit of output recording.  For example,
one can draw with \cl{+foreground-ink+}, change to a different
\cl{medium-foreground}, and replay the output record; the replayed output will
come out in the new color.

You can change the foreground or background design of a medium at any time.
This changes the contents of the medium's drawing plane.  The effect is as if
everything on the drawing plane is erased, the background design is drawn onto
the drawing plane, and then everything that was ever drawn (provided it was
saved in the output history) is drawn over again, using the medium's new
foreground and background.

If an infinite recursion is created using an indirect ink, an error is
signalled when the recursion is created, when the design is used for drawing,
or both.

Two indirect inks are defined, but no advertised way is provided to create more
of them.

\Defconst {+foreground-ink+}

An indirect ink that uses the medium's foreground design.

\Defconst {+background-ink+}

An indirect ink that uses the medium's background design.


\section {Flipping Ink}

\Defun {make-flipping-ink} {design1 design2}

Returns a design that interchanges occurrences of the two \term{designs}
\arg{design1} and \arg{design2}.  Drawing the resulting design over a background
(either by drawing or with \cl{compose-over}) is defined to change the color in
the background that would have been drawn by \arg{design1} at that point into
the color that would have been drawn by \arg{design2} at that point, and vice
versa.  The effect on any color other than the colors determined by those two
designs is unspecified; however, drawing the same figure twice using the same
flipping ink is guaranteed to be an ``identity'' operation.  If either
\arg{design1} or \arg{design2} is not solid, the consequences are unspecified.
The purpose of flipping is to allow the use of ``XOR hacks'' for temporary
changes to the display.

The opacity of a flipping ink is zero at points where the opacity of either
\arg{design1} or \arg{design2} is zero.  Otherwise the opacity of a flipping ink
is 1.  If \arg{design1} or \arg{design2} is translucent, the consequences are
unspecified.  If \cl{compose-in} or \cl{compose-out} is used to make a flipping
ink translucent, the consequences are unspecified.

If \arg{design1} and \arg{design2} are equivalent, the result can be
\cl{+nowhere+}.

In Release 2, \cl{make-flipping-ink} might require \arg{design1} and
\arg{design2} to be colors.


\Defconst {+flipping-ink+}

A flipping ink that flips \cl{+foreground-ink+} and \cl{+background-ink+}.


\section {Examples of Simple Drawing Effects}

\paragraph {Drawing in the foreground color.}

Use the default, or specify \cl{:ink +foreground-ink+}.

\paragraph {Erasing.}

Specify \cl{:ink +background-ink+}.

\paragraph {Drawing in color.}

Specify \cl{:ink +green+}, \cl{:ink (make-rgb-color 0.6 0.0 0.4)}, and so forth.

\paragraph {Drawing an opaque gray.}

Specify \cl{:ink (make-gray-color 0.25)} to draw in a shade of gray independent
of the window's foreground color.  On a non-color, non-grayscale display this
will generally turn into a stipple.
