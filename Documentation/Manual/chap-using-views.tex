\chapter{Using views}

The CLIM specification mentions a concept called a \emph{view}, and
also lists a number of predefined views to be used in various
different contexts.

In this chapter we show how the \emph{view} concept can be used in
some concrete programming examples.  In particular, we show how to use
a single pane to show different views of the application data
structure at different times.  To switch between the different views,
we supply a set of commands that alter the
\texttt{stream-default-view} feature of all CLIM extended output
streams.

The example shown here has been stripped to a bare minimum in order to
illustrate the important concepts.  A more complete version can be
found in \texttt{Examples/views.lisp} in the McCLIM source tree.

Here is the example:

\verbatiminput{views.lisp}

The example shows a stripped-down example of a simple database of
members of some organization.

The main trick used in this example is the \texttt{display-main-pane}
function that is declared to be the display function of the main pane
in the application frame.  The \texttt{display-main-pane} function
trampolines to a generic function called
\texttt{display-pane-with-view}, and which takes an additional
argument compared to the display functions of CLIM panes.  This
additional argument is of type \texttt{view} which allows us to
dispatch not only on the type of frame and the type of pane, but also
on the type of the current default view.  In this example the view
argument is simply taken from the default view of the pane.

A possibility that is not obvious from reading the CLIM specification
is to have views that contain additional slots.  Our example defines
two subclasses of the CLIM \texttt{view} class, namely
\texttt{members-view} and \texttt{person-view}.

The first one of these does not contain any additional slots, and is
used when a global view of the members of our organization is wanted.
Since no instance-specific data is required in this view, we follow
the idea of the examples of the CLIM specification to instantiate a
singleton of this class and store that singleton in the
\texttt{stream-default-view} of our main pane whenever a global view
of our organization is required.

The \texttt{person-view} class, on the other hand, is used when we
want a closer view of a single member of the organization.  This class
therefore contains an additional slot which holds the particular
person instance we are interested in.  The method on
\texttt{display-pane-with-view} that specializes on
\texttt{person-view} displays the data of the particular person that
is contained in the view.

To switch between the views, we provide two commands.  The command
\texttt{com-show-all} simply changes the default view of the main pane
to be the singleton instance of the \texttt{members-view} class.  The
command \texttt{com-show-person} is more complicated.  It takes an
argument of type person, creates an instance of the
\texttt{person-view} class initialized with the person that was passed
as an argument, and stores the instance as the default view of the
main pane.

