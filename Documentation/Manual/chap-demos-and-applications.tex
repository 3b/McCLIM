\chapter{CLIM demos and applications}

\section{Running the demos}

The McCLIM source distribution comes with a number of demos and
applications.  They are intended to showcase specific CLIM features,
demonstrate programming techniques or provide useful tools.

These demos and applications are available in the \texttt{Examples}
and \texttt{Apps} subdirectories of the source tree's root directory.
Instructions for compiling, loading and running some of the demos are
included in the file \texttt{INSTALL} with the McCLIM installation
instructions for your Common Lisp implementation.

Below is a complete list of the McCLIM demos and applications, sorted in
alphabetical order.  Each entry provides a short description of what the
program does, with instructions for compiling and running it if not
mentioned in the general installation instructions.

\begin{itemize}
\item Apps/Listener
CLIM-enabled Lisp listener.  See the compilation and execution
instructions in \texttt{Apps/Listener/README}.
\item Apps/Inspector
CLIM-enabled Lisp inspector.  See the compilation and execution
instructions in \texttt{Apps/Inspector/INSTALL}.
\item Examples/address-book.lisp
Simple address book.  See McCLIM's installation instructions.
\item Examples/calculator.lisp
Simple desk calculator.  See McCLIM's installation instructions.
\item Examples/clim-fig.lisp
Simple paint program.  You can run it by evaluating this form at the
Lisp prompt:
\begin{verbatim}
(clim-demo::clim-fig)
\end{verbatim}
\item Examples/colorslider.lisp
Interactive color editor.  See McCLIM's installation instructions.
\item Examples/demodemo.lisp
Demonstrates different pane types.  You can compile it by evaluating:
\begin{verbatim}
(compile-file "Examples/demodemo.lisp")
\end{verbatim}
Then load it with:
\begin{verbatim}
(load "Examples/demodemo")
\end{verbatim}
Finally, run it with:
\begin{verbatim}
(clim-demo::demodemo)
\end{verbatim}
\item Examples/goatee-test.lisp
Text editor with Emacs-like key bindings.  See McCLIM's installation
instructions.
\item Examples/menutest.lisp
Displays a window with a simple menu bar.  See McCLIM's installation
instructions.
\item Examples/postscript-test.lisp
Displays text and graphics to a PostScript file.  Run it with:
\begin{verbatim}
(clim-demo::postscript-test)
\end{verbatim}
The resulting file \texttt{ps-test.ps} is generated in the current directory
and can be displayed by a PostScript viewer such as \texttt{gv} on Unix-like
systems.
\item Examples/presentation-test.lisp
Displays an interactive window in which you type numbers that are
successively added.  When a number is expected as input, you can either
type it at the keyboard, or click on a previously entered number. Run it
with:
\begin{verbatim}
(clim:run-frame-top-level (clim:make-application-frame
                           'clim-demo::summation))
\end{verbatim}

\item Examples/sliderdemo.lisp
Apparently a calculator demo (see above).  Compile with:
\begin{verbatim}
(compile-file "Examples/sliderdemo.lisp")
\end{verbatim}
Load with:
\begin{verbatim}
(load "Examples/sliderdemo")
\end{verbatim}
Run with:
\begin{verbatim}
(clim-demo::slidertest)
\end{verbatim}
\item Examples/stream-test.lisp
Interactive command processor that echoes its input.  Run with:
\begin{verbatim}
(clim-demo::run-test)
\end{verbatim}
\end{itemize}

The following programs are currently \textbf{known not to work}:
\begin{itemize}
\item
 \texttt{Examples/fire.lisp}
\item
 \texttt{Examples/gadget-test-kr.lisp}
\item
 \texttt{Examples/gadget-test.lisp}
\item
 \texttt{Examples/puzzle.lisp}
\item
 \texttt{Examples/traffic-lights.lisp}
\item
 \texttt{Examples/transformations-test.lisp}
\end{itemize}

\section{McCLIM installation and usage tips}

This section collects useful installation and usage tips.  They refer to
specific Common Lisp implementations or McCLIM features.

\subsection{Multiprocessing with CMUCL}

Before beginning a McCLIM session with CMUCL, \textbf{you are strongly
advised} to initialize multiprocessing by evaluating the form:
\begin{verbatim}
(mp::startup-idle-and-top-level-loops)
\end{verbatim}
If you use the SLIME development environment under Emacs, evaluate the
above form from the \texttt{*inferior-lisp*} buffer, not from
\texttt{*slime-repl[n]*}.  Initializing multiprocessing can make a difference
between an application that starts instantaneously on, say, a Pentium IV
class PC, and one that may take \emph{minutes} on the same machine.


\subsection{Adding mouse button icons}

McCLIM comes with experimental code for adding graphical mouse button
icons to pointer documentation panes.  To use this feature, you have to
first compile the file \texttt{Experimental/pointer-doc-hack.lisp} in the
source tree.  Assuming you have built McCLIM from source as explained in
the installation instructions, evaluate this form to compile the file:
\begin{verbatim}
(compile-file "Experimental/pointer-doc-hack.lisp")
\end{verbatim}
Then, to activate the feature, load the compiled file before starting a
McCLIM session or application:
\begin{verbatim}
(load "Experimental/pointer-doc-hack")
\end{verbatim}
Alternatively, you may dump a Lisp image containing McCLIM and the
graphical pointer documentation code.  See the documentation of your
Common Lisp system for more information.
