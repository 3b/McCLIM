\chapter{Menu bar}
\index{menu-bar}

Menu bar has become essential part of every GUI system, including McClim. Ideally, McClim should try to use the menu bar provided by host window system via McClim backends, but the current \texttt{clx-backend} doesn't supports native menu bars. That's why It has some quirks of its own, like you need to keep mouse button pressed while accessing the sub-menus.

\section{Creating Menu bar}
\label{creating-menu-bar}

McClim makes creating menu bar quite easy.

\begin{verbatim}
        (define-application-frame foo ()
        ...
        (:menu-bar t)
        ...)
\end{verbatim}

Options for \texttt{:menu-bar} can be
\begin{itemize}
\item
  \texttt{t} (default) : McClim will provide the menu bar. Later, when you start defining commands, you can provide a \texttt{(:menu t)} argument to command definition that will add this command to menu bar.
\item
  \texttt{nil} : McClim won’t provide the menu bar.
\item
  \texttt{command-table} : If you provide a named command table as argument, that command table is used to provide the menu bar(See Chapter \ref{using-command-tables} on command tables). 
\end{itemize}

To add a sub-menu to menu bar, You need to change the type of menu-item from \texttt{:command} to \texttt{:menu} (which requires another \texttt{command-table} as argument) (See \refSec{modifying-menu-items}). 

\section{Modifying Menu bar}

Menu bar can be changed anytime by changing \texttt{command-table} associated with the current \texttt{frame}. For example:
\begin{verbatim}
        (setf (frame-command-table *application-frame*) 
              new-command-table)
\end{verbatim}
changes menu bar of \texttt{*application-frame*} by replacing current \texttt{command-table} (accessible by \texttt{frame-command-table} function) with \texttt{new-command-table}.

\subsection{Modifying menu items of command table}
\label{modifying-menu-items}

Menu items can be added to command table with following function:

\Defun {add-menu-item-to-command-table} {command-table string type value
  \\\key documentation (after ':end) keystroke text-style (errorp t)}

Arguments to this function are:
\begin{itemize}
\item
  \texttt{command-table} : Command table to which we want to add the menu item.
\item
  \texttt{string} : name of the menu item as it will appear on the menu bar. Its character case is ignored e.g. you may give it \texttt{file} or \texttt{FILE} but it will appear as \texttt{File}.
\item
  \texttt{type} and \texttt{value} : type could be one of \texttt{:command},\texttt{:function},\texttt{:menu} and \texttt{:divider}. value of \texttt{value} depends on \texttt{type}. So when given the 
  \begin{itemize}
  \item
    \texttt{:command} : \texttt{value} must be a command, a cons of command name and it's arguments. if you omit the arguments McClim will prompt for arguments.
  \item
    \texttt{:function} : \texttt{value} must be a function having indefinite extent that, when called, returns a command. function must accep two arguments, the gesture (keyboard or mouse press event) and a ``numeric argument''.
  \item
    \texttt{:menu} : \texttt{value} must be another command table. This type is used to add sub-menus to the menu.
  \item
    \texttt{:divider} : \texttt{value} is ignored, and \texttt{string} is used as a divider string. Using ``|'' as string will make it obvious to users that it is a divider.
  \item
    \texttt{documentation} : You can provide the documentation (for non-obvious menu items) which will be displayed on pointer-documentation pane (if you have one).
  \item
    \texttt{after} (default \texttt{:end}) : This determines where item will be inserted in the menu. The default is to add it to the end. Other values could be \texttt{:start}, \texttt{:sort}(add in alphabetical order) or \texttt{string} which is name of existing menu-item to add after it.
  \item
    \texttt{keystroke} : If keystroke is supplied, it will be added to comand tables keystroke accelerator table. Value must be a keyboard gesture name e.g. (:s :control) for Control + s.
  \item
    \texttt{text-style} : This is either a text style spec or nil. It is used to indicate that the command menu item should be drawn with the supplied text style in command menus.
  \item
    \texttt{error-p} : If this is \texttt{t}, adding an existing item to the menu will signal error. If \texttt{nil}, it will first remove the existing item and add the new item to the command-table. 
  \end{itemize}
\end{itemize}

To remove items from command table, following function is used:

\Defun {remove-menu-item-from-command-table} {command-table string \key (errorp t)}

where \texttt{command-table} is command-table-designator and \texttt{string} is menu item's name (it is case-insensitive). You can provide \texttt{:error-p nil} to suppress the error if item is not in the command-table.

Note that both of above functions \texttt{\emph{does not}} automatically update the menu bar. For that you need to replace existing \texttt{frame-command-table} with modified command table using \texttt{setf}. Ideal way to do this is use \texttt{let} to create the copy of \texttt{frame-command-table}, modify it and at the end call \texttt{setf} to replace the original.





