\chapter{Pane names}
\label{chap-spec-issue-pane-names}

There are several places in the specification where \emph{pane names}
are mentioned.

\section{Pane initialization and pane properties}
\label{sec-spec-issue-pane-name-pane-initialization-and-pane-properties}

The first place is in sections 29.2.1 (Pane Initialization Options).
In this section, we learn that the keyword argument \texttt{:name} can
be used as an initialization option to the function \texttt{make-pane}
and to the generic function \texttt{make-pane-1}, and that the default
value of this option is \texttt{nil}.  There is no indication
regarding the \emph{type} that the value of this option can take, but
all the examples used in the specification use symbols for this name.
This section also mentions that the \texttt{:name} initialization
option must be accepted by all pane classes.

The second place where \emph{pane names} are mentioned is in section
29.2.2 (Pane Properties).  In this section, we learn that the generic
function named \texttt{pane-name} returns the name of the pane.  The
is no corresponding \texttt{setf} function, indicating that the name
of the pane is immutable.

In other parts of the specification, there are hints that indicate
that panes may be unnamed.  Presumably, when \texttt{nil} is the value
of the initialization option, this indicates that the pane is unnamed,
but this fact is nowhere explicitly mentioned in the specification.

\section{Application frame functions}

The third place where \emph{pane names} are mentioned is in section
28.3 (Application Frame Functions).  There ares several occurrences of
pane names in this section

\subsection{\texttt{frame-standard-output}}

The description of the generic function \texttt{frame-standard-output}
says that the default method returns the first named pane of type
\texttt{application-pane} that is visible in the current layout, and
that if there is no such pane, then it returns the first pane of type
\texttt{interactor-pane} that is exposed in the current layout.

First of all, the description of this function does not mention what
is returned if there is no pane that fits this description.

More importantly, one may wonder why the candidate application pane
has to be named.  As we mentioned in
\refSec{sec-spec-issue-pane-name-pane-initialization-and-pane-properties},
presumably a pane is considered unnamed when its name was initialized
to \texttt{nil}.  According to the specification, an unnamed
application pane does not qualify, and if no named application pane
fitting the description can be found, then instead an interactor pane
is chosen.  According to the specification, that interactor pane does
not have to be named, however.

\subsection{\texttt{frame-standard-input}}

The description of the generic function \texttt{frame-standard-input}
says that the default method returns the first named pane of type
\texttt{interactor-pane} that is visible in the current layout, and
that if there is no such pane, then it returns the value returned by a
call to \texttt{frame-standard-output}.

As a consequence of this rule, it appears that it is preferable to
have a named pane than to have a pane of type
\texttt{interactor-pane}.

\subsection{\texttt{frame-panes}}

According to the specification, this generic function return the pane
that is the top-level pane in the current layout of the named panes of
the frame given as an argument.  It is hard to understand what is
meant by the restriction regarding the names here.  Presumably, it is
possible for some children of the top-level pane to be unnamed.

\subsection{\texttt{frame-current-panes}}

According to the specification, this generic function returns a list
of the named panes in the current layout of the frame given as an
argument.  And if there are no named panes, then the specification
says that only the single, top-level pane is returned.

It is not clear whether in the second case, a \emph{list} containing
that single pane is returned, or whether that pane is returned as an
atom.

\subsection{\texttt{get-frame-pane}}

According to the specification, this generic function returns the
named CLIM stream pane with the name given as an argument.  Again, no
indication is given as to the nature of the name, nor what equality
function might be used to determine whether the right name has been
found.  And, no indication is given as to what happens if there is no
pane with the name given.

\subsection{\texttt{find-pane-named}}

According to the specification, this generic function returns the
pane with the name given as argument.  The difference between this
function and the function \texttt{get-frame-pane} is that this
function can return a pane of any type, and not just a CLIM stream
pane.  Otherwise, the same omissions apply.

\subsection{\texttt{redisplay-frame-pane}}

This generic function is called in order to redisplay a particular
pane of a frame.  The \textit{pane} argument can be a pane object or
the name of a pane.
