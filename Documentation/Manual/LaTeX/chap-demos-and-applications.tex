\chapter{CLIM demos and applications}

\section{Running the demos}

The McCLIM source distribution comes with a number of demos and
applications.  They are intended to showcase specific CLIM features,
demonstrate programming techniques or provide useful tools.

These demos and applications are available in the \texttt{Examples}
and \texttt{Apps} subdirectories of the source tree's root directory.
Instructions for compiling, loading and running some of the demos are
included in the file \texttt{INSTALL} with the McCLIM installation
instructions for your Common Lisp implementation.

Below is a complete list of the McCLIM demos and applications, sorted in
alphabetical order.  Each entry provides a short description of what the
program does, with instructions for compiling and running it if not
mentioned in the general installation instructions.

\subsection{Applications}

\begin{itemize}
\item Apps/Listener

  CLIM-enabled Lisp listener. System name is
  \texttt{clim-listener}. See instructions in
  \texttt{Apps/Listener/README} for more information.

\item Apps/Inspector

  CLIM-enabled Lisp inspector. System name is \texttt{clouseau}. See
  instructions in \texttt{Apps/Inspector/INSTALL} for more
  information..

\item Apps/Debugger

  Common Lisp debugger implemented in McCLIM. It uses the portable
  debugger interface developed for the Slime project. Application has
  some quirks and requires work. System name is
  \texttt{clim-debugger}.

\item Apps/Functional-Geometry

  Peter Henderson idea, see http://www.ecs.soton.ac.uk/~ph/funcgeo.pdf
  and http://www.ecs.soton.ac.uk/~ph/papers/funcgeo2.pdf implemented
  in Lisp by Frank Buss. CLIM Listener interface by Rainer
  Joswig. System name is \texttt{functional-geometry}.

\begin{verbatim}
(functional-geometry:run-functional-geometry)
(clim-plot *fishes*) ; from a listener
\end{verbatim}

\item Apps/Scigraph

  Scigraph Scientific Graphing Package. See the compilation and
  execution instructions in \texttt{Apps/Scigraph/README}.
\end{itemize}

\subsection{Demos and tests}

Demos are meant to be run after loading the \texttt{clim-examples}
system from the frame created with \texttt{(clim-demo:demodemo)}.

\begin{verbatim}
(asdf:load-system 'clim-examples)
(clim-demo:demodemo)
\end{verbatim}

Available demos and tests are defined in the following files:

\begin{itemize}
\item Examples/demodemo.lisp

  Demonstrates different pane types and other tests.

\item Examples/clim-fig.lisp

  Simple paint program.

\item Examples/calculator.lisp

  Simple desk calculator.

\item Examples/method-browser.lisp

  Example of how to write a CLIM application with a ``normal'' GUI,
  where ``normal'' is a completely event driven app built using
  gadgets and not using the command-oriented framework.

\item Examples/address-book.lisp

  Simple address book.

\item Examples/puzzle.lisp

  Simple puzzle game.

\item Examples/colorslider.lisp

  Interactive color editor.

\item Examples/logic-cube.lisp

  Software-rendered 3d logic cube game. Shows how the transformations
  work and how to implement custom handle-repaint methods.

\item Examples/menutest.lisp

  Displays a window with a simple menu bar.

\item Examples/gadget-test.lisp

  Displays a window with various gadgets.

\item Examples/dragndrop.lisp

  Example of ``Drag and Drop'' functionality.

\item Examples/dragndrop-translator.lisp

  Another example of ``Drag and Drop'' functionality (with colors!).

\item Examples/draggable-graph.lisp

  Demo of draggable graph nodes.

\item Examples/image-viewer.lisp

  A simple program for displaying images of formats known to McCLIM.

\item Examples/font-selection.lisp

  A font selection dialog.

\item Examples/tabdemo.lisp

  A tab layout demo (McCLIM extension).

\item Examples/postscript-test.lisp

  Displays text and graphics to a PostScript file.  Run it with:

\begin{verbatim}
(clim-demo::postscript-test)
\end{verbatim}

  The resulting file \texttt{ps-test.ps} is generated in the current
  directory and can be displayed by a PostScript viewer such as
  \texttt{gv} on Unix-like systems.

\item Examples/presentation-test.lisp

  Displays an interactive window in which you type numbers that are
  successively added.  When a number is expected as input, you can
  either type it at the keyboard, or click on a previously entered
  number. Labeled ``Summation''.

\item Examples/sliderdemo.lisp

  Apparently a calculator demo (see above). Labeled ``Slider demo''.

\item Examples/stream-test.lisp

  Interactive command processor that echoes its input in
  \texttt{*trace-output*}.

\end{itemize}

The following programs are currently \textbf{known not to work}:
\begin{itemize}
\item
 \texttt{Examples/gadget-test-kr.lisp}
\item
 \texttt{Examples/traffic-lights.lisp}
\end{itemize}

\section{McCLIM installation and usage tips}

This section collects useful installation and usage tips.  They refer to
specific Common Lisp implementations or McCLIM features.

\subsection{Adding mouse button icons}

McCLIM comes with experimental code for adding graphical mouse button
icons to pointer documentation panes.  To use this feature, you have to
first compile the file \texttt{Experimental/pointer-doc-hack.lisp} in the
source tree.  Assuming you have built McCLIM from source as explained in
the installation instructions, evaluate this form to compile the file:
\begin{verbatim}
(compile-file "Experimental/pointer-doc-hack.lisp")
\end{verbatim}
Then, to activate the feature, load the compiled file before starting a
McCLIM session or application:
\begin{verbatim}
(load "Experimental/pointer-doc-hack")
\end{verbatim}
Alternatively, you may dump a Lisp image containing McCLIM and the
graphical pointer documentation code.  See the documentation of your
Common Lisp system for more information.
