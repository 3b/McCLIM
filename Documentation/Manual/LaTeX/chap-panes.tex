\chapter{Panes}
\index{pane}

Panes are subclasses of sheets.  Some panes are \emph{layout panes}
that determine the size and position of its children according to rules
specific to each particular type of layout pane.  Examples of layout
panes are vertical and horizontal boxes, tables etc.

According to the CLIM specification, all CLIM panes are
\emph{rectangular objects}.  For McCLIM, we interpret that phrase to
mean that:

\begin{itemize}
\item
 CLIM panes appear rectangular in the native windowing system;
\item
 CLIM panes have a native transformation that does not have a rotation
  component, only translation and scaling.
\end{itemize}

Of course, the specification is unclear here.  Panes are subclasses of
sheets, and sheets don't have a shape per-se.  Their \emph{regions} may
have a shape, but the sheet itself certainly does not.

The phrase in the specification \emph{could} mean that the
\emph{sheet-region} of a pane is a subclass of the region class
\emph{rectangle}.  But that would not exclude the possibility that the
region of a pane would be some non-rectangular shape in the
\emph{native coordinate system}.  For that to happen, it would be
enough that the \emph{sheet-transformation} of some ancestor of the
pane contain a rotation component.  In that case, the layout protocol
would be insufficient in its current version.

McCLIM panes have the following additional restrictions:

\begin{itemize}
\item
 McCLIM panes have a coordinate system that is only a translation
  compared to that of the frame manager;
\item
 The parent of a pane is either nil or another pane.
\end{itemize}

Thus, the panes form a \emph{prefix} in the hierarchy of sheets.  It is
an error for a non-pane to adopt a pane.

Notice that the native transformation of a pane need not be the identity
transformation.  If the pane is not mirrored, then its native
transformation is probably a translation of that of its parent.

Notice also that the native transformation of a pane need not be the
composition of the identity transformation and a translation.  That
would be the case only of the native transformation of the top level
sheet is the identity transformation, but that need not be the case.  It
is possible for the frame manager to impose a coordinate system in (say)
millimeters as opposed to pixels.  The native transformation of the top
level sheet of such a frame manager is a scaling with coefficients other
than 1.

\section{Creating panes}
\label{sec-panes-creating}

There is some confusion about the options that are allowed when a pane
is created with \texttt{make-pane}.  Some parts of the specification
suggest that stream panes such as application panes and interactor
panes can be created using \texttt{make-pane} and an option
\texttt{:scroll-bars}.  Since these application panes do not in
themselves contain any scroll bars, using that option results in a
pane hierarchy being created with the topmost pane being a pane of
type \texttt{scroller-pane}.

As far as McCLIM is concerned, this option to \texttt{make-pane} is
obsolete.%
\footnote{In the specification, there is no example of the use of this
  option to \texttt{make-pane} or to the equivalent keywords in the
  \texttt{:panes} section of \texttt{define-application-frame}.  There
  is however one instance where the \texttt{:scroll-bars} option is
  mention for pane creation.  We consider this to be an error in the
  specification.}
The same goes for using this option together with the equivalent
keyword, i.e., \texttt{:application} or \texttt{interactor}, in the
\texttt{:panes} section of \texttt{define-application-frame}.

Instead, we recommend following the examples of the specification,
where scroll bars are added in the \texttt{layouts} section of
\texttt{define-application-frame}.

\index{Panes order}

When specification talks about panes in a fashion implying some order
(i.e ``first application-pane'') McCLIM assumes order of definition,
not order of appearing in layout. Particularly that means, that if one
pane is put before another in \texttt{:PANES} option, then it precedes
it. It is relevant to \texttt{FRAME-STANDARD-OUTPUT} (therefore
binding of \texttt{*STANDARD-OUTPUT*}) and other similar functions.

\section{Pane names}

Every pane class accepts the initialization argument \texttt{:name}
the value of which is typically a symbol in the package defined by the
application.  The generic function \texttt{pane-name} returns the
value of this initialization argument.  There is no standard way of
changing the name of an existing pane.  Using the function
\texttt{reinitialize-instance} may not have the desired effect, since
the application frame may create a dictionary mapping names to panes,
and there is no way to invalidate the contents of such a potential
dictionary.

The function \texttt{find-pane-named} searches the pane hierarchy of
the application frame, consulting the names of each pane until a
matching name is found.  The CLIM specification does not say what
happens if a name is given that does not correspond to any pane.
McCLIM returns \texttt{nil} in that case.  If pane names are not
unique, it is unspecified which of several panes is returned by a call
to this function.

If the advice of \refSec{sec-panes-creating} is followed, then the
name given in the \texttt{:panes} option of the macro
\texttt{define-application-frame} will always be the name of the
top-level pane returned by the \textit{body} following the pane name.

If that advice is not followed, then the name given to a pane in the
\texttt{:panes} option of the macro \texttt{define-application-frame}
may or may not become the name of the pane that is constructed by the
\textit{body} that follows the name.  Recall that the syntax of the
expression that defines a pane in the \texttt{:panes} option is
\texttt{(}\textit{name} \texttt{.} \textit{body}\texttt{)}.
Currently, McCLIM does the following:

\begin{itemize}
\item If the \textit{body} creates a pane by using a keyword, or by
  using an explicitly mentioned call to \texttt{make-pane}, then the
  name is given to the pane of the type explicitly mentioned, even
  when the option \texttt{:scroll-bars} is given.
\item If the \textit{body} creates a pane by calling some arbitrary
  form other than a call to \texttt{make-pane}, then the name is
  given to the topmost pane returned by the evaluation of that form.
\end{itemize}

We reserve the right to modify this behavior in the future.
Application code should respect the advice given in
\refSec{sec-panes-creating}.

\section{Redisplaying panes}

Recall that \emph{redisplay} refers to the creation of the output
history of a pane.  There are two typical ways of creating this output
history:

\begin{itemize}
\item The application maintains some data structure that needs to be
  reflected in the text and graphics of the pane.  In this case,
  a pane of type \texttt{application-pane} is typically used, and the
  default value of the \texttt{:display-time} option is used, which
  means that some kind of application-supplied \emph{display function}
  is executed at the end of each iteration of the command loop.  In
  this situation, the output history is either recomputed from scratch
  in each iteration, or the programmer can use the \emph{incremental
    redisplay} facility to reuse some of the existing output records
  in the history.
\item The application does not keep any data structure, and instead
  generates output incrementally, either as a result of some user
  action, or of some data arriving from an external source.  In this
  case, the \texttt{:display-time} option is either going to be
  \texttt{t} or \texttt{nil}.  With both of these options, the output
  history is maintained intact after each iteration of the command
  loop.  Instead, when user actions are issued, more output records
  are simply added to the existing output history.
\end{itemize}

For the second possibility, the pane is never redisplayed.  Instead,
the action of updating the pane contents is referred to as
\emph{replaying} the output history.  The remainder of this section is
entirely dedicated to the \emph{redisplay} action.

It is occasionally necessary for the application to redisplay a pane
explicitly, as opposed to letting the command loop handle it.  For
example, if the application data structure is updated in some way, but
this update is not the result of a command, then after such an update,
the redisplay function needs to be executed explicitly.  Such an
update could be the result of a timer event, or of communication with
an external process.

\Defgeneric {redisplay-frame-pane} {frame pane \key force-p}

Calling this generic function causes an immediate redisplay of
\textit{pane}.  When \textit{force-p} is false and the incremental
redisplay facility is in use for \textit{pane}, then output records are
reused as appropriate.  Supplying a true value for \textit{force-p}
causes the entire output history to be recomputed from scratch.

Notice that this function does not check whether the pane has been
marked to need redisplay, as indicated by a call to the generic
function \texttt{pane-needs-redisplay}.  It results in an
unconditional redisplay of \textit{pane}.

\Defgeneric {redisplay-frame-panes} {frame \key force-p}

Calling this generic function causes an immediate redisplay of all the
panes of \textit{frame} that are visible in the current layout.  This
function simply calls \texttt{redisplay-frame-pane} for each visible
pane of \textit{frame}.

Again, notice that no check is being made as to whether the visible
panes have been marked as needing redisplay.  This function calls
\texttt{redisplay-frame-pane} unconditionally for each visible pane,
and since \texttt{redisplay-frame-pane} redisplays the pane
unconditionally, it follows that all visible panes are unconditionally
redisplayed.

Also notice that the implication of this unconditional behavior on the
part of \texttt{redisplay-frame-panes} means that this is not the
function called by the standard command loop.  The standard command
loop only redisplays panes that have been marked as needing redisplay,
though when the value of the \texttt{:display-time} option is
\texttt{:command-loop} for some pane, then it is always marked as
needing redisplay in each iteration of the command loop.

\section{Layout protocol}
\index{layout protocol}

There is a set of fundamental rules of CLIM dividing responsibility
between a parent pane and a child pane, with respect to the size and
position of the region of the child and the \emph{sheet transformation}
of the child.  This set of rules is called the \emph{layout protocol}.

The layout protocol is executed in two phases.  The first phase is
called the \emph{space compostion} phase, and the second phase is
called the \emph{space allocation} phase.

\subsection{Space composition}

The space composition is accomplished by the generic function
\texttt{compose-space}.  When applied to a pane, \texttt{compose-space}
returns an object of type \emph{space-requirement} indicating the needs
of the pane in terms of preferred size, minimum size and maximum size.
The phase starts when compose-space is applied to the top-level pane of
the application frame.  That pane in turn may ask its children for their
space requirements, and so on until the leaves are reached.  When the
top-level pane has computed its space requirments, it asks the system
for that much space.  A conforming window manager should respect the
request (space wanted, min space, max space) and allocate a top-level
window of an acceptable size.  The space given by the system must then
be distributed among the panes in the hierarchy
\refSec{space-allocation}.

Each type of pane is responsible for a different method on
\texttt{compose-space}.  Leaf panes such as \emph{labelled gadgets} may
compute space requirements based on the size and the text-style of the
label.  Other panes such as the vbox layout pane compute the space as a
combination of the space requirements of their children.  The result of
such a query (in the form of a space-requirement object) is stored in
the pane for later use, and is only changed as a result of a call to
\texttt{note-space-requirement-changed}.

Most \emph{composite panes} can be given explicit values for the values
of \texttt{:width}, \texttt{:min-width}, \texttt{:max-width},
\texttt{:height}, \texttt{:min-height}, and \texttt{:max-height}
options.  If such arguments are not given (effectively making these
values nil), a general method is used, such as computing from children
or, for leaf panes with no such reasonable default rule, a fixed value
is given.  If such arguments are given, their values are used instead.
Notice that one of \texttt{:height} and \texttt{:width} might be
given, applying the rule only in one of the dimensions.

Subsequent calls to \texttt{compose-space} with the same arguments are
assumed to return the same space-requirement object, unless a call to
note-space-requirement-changed has been called in between.

\subsection{Space allocation}
\label{space-allocation}

When \texttt{allocate-space} is called on a pane \texttt{P}, it must
compare the space-requirement of the children of \texttt{P} to the
available space, in order to distribute it in the most preferable way.
In order to avoid a second recursive invokation of
\texttt{compose-space} at this point, we store the result of the
previous call to \texttt{compose-space} in each pane.

To handle this situtation and also explicitly given size options, we use
an \texttt{:around} method on \texttt{compose-space}.  The
\texttt{:around} method will call the primary method only if necessary
(i.e., \texttt{(eq (slot-value pane 'space-requirement) nil)}), and store
the result of the call to the primary method in the
\texttt{space-requirement} slot.

We then compute the space requirement of the pane as follows:

\begin{verbatim}
     (setf (space-requirement-width ...)  (or explicit-width
           (space-requirement-width request)) ...
           (space-requirement-max-width ...)  (or explicit-max-width
           explicit-width (space-requirement-max-width request)) ...)
\end{verbatim}

When the call to the primary method is not necessary we simply return
the stored value.

The \texttt{spacer-pane} is an exception to the rule indicated above.  The
explicit size you can give for this pane should represent the margin
size.  So its primary method should only call compose on the child.  And
the around method will compute the explicit sizes for it from the space
requirement of the child and for the values given for the surrounding
space.

\subsection{Change-space Notification Protocol}

The purpose of the change-space notification protocol is to force a
recalculation of the space occupied by potentially each pane in the
\emph{pane hierarchy}.  The protocol is triggerred by a call to
\texttt{note-space-requirement-changed} on a pane \texttt{P}. In McCLIM, we
must therefore invalidate the stored space-requirement value and
re-invoke \texttt{compose-space} on \texttt{P}.  Finally, the
\emph{parent} of \texttt{P} must be notified recursively.

This process would be repeated for all the panes on a path from \texttt{P}
to the top-level pane, if it weren't for the fact that some panes
compute their space requirements independently of those of their
children.  Thus, we stop calling \texttt{note-space-requirement-changed}
in the following cases:

\begin{itemize}
\item
 when \texttt{P} is a \texttt{restraining-pane},
\item
 when \texttt{P} is a \texttt{top-level-sheet-pane}, or
\item
 when \texttt{P} has been given explicit values for \texttt{:width} and
  \texttt{:height}
\end{itemize}

In either of those cases, \texttt{allocate-space} is called.
