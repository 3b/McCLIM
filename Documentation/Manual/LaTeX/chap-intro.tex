\chapter{Introduction}
\pagenumbering{arabic}%

CLIM is a large layered software system that allows the user to
customize it at each level.  The most simple ways of using CLIM is to
directly use its top layer, which contains application frames, panes,
and gadgets, very similar to those of traditional windowing system
toolkits such as GTK, Tk, and Motif.

But there is much more to using CLIM.  In CLIM, the upper layer with
panes and gadgets is written on top of a basic layer containing more
basic functionality in the form of sheets.  Objects in the upper layer
are typically instances of classes derived from those of the lower
layer.  Thus, nothing prevents a user from adding new gadgets and panes
by writing code that uses the sheet layer.

Finally, since CLIM is written in Common Lisp, essentially all parts of
it can be modified, replaced, or extended.

For that reason, a user's manual for CLIM must contain not only a
description of the protocols of the upper layer, but also of all
protocols, classes, functions, macros, etc. that are part of the
specification.

\section{Standards}

This manual documents McCLIM which is a mostly
complete implementation of the CLIM 2.0 specification and its revision
2.2. To our knowledge version~2.2 of the CLIM specification is only
documented in the ``CLIM 2 User's Guide'' by Franz. While that document
is not a formal specification, it does contain many cleanups and is
often clearer than the official specification; on the other hand, the
original specification is a useful reference. This manual will note
where McCLIM has followed the 2.2 API.

Also, some protocols mentioned in the 2.0 specification, such as parts
of the incremental redisplay protocol, are clearly internal to CLIM and
not well described.  It will be noted here when they are partially
implemented in McCLIM or not implemented at all.

\section{How CLIM is different}

Many new users of CLIM have a hard time trying to understand how it
works and how to use it.  A large part of the problem is that many such
users are used to more traditional GUI toolkits, and they try to fit
CLIM into their mental model of how GUI toolkits should work.

But CLIM is much more than just a GUI toolkit, as suggested by its name,
it is an \emph{interface manager}, i.e. it is a complete mediator
between application ``business logic'' and the way the user interacts
with objects of the application.  In fact, CLIM doesn't have to be used
with graphics output at all, as it contains a large collection of
functionality to manage text.

Traditional GUI toolkits have an \emph{event loop}.
Events are delivered to GUI elements called \emph{gadgets} (or
\emph{widgets}), and the programmer attaches \emph{event handlers} to
those gadgets in order to invoke the functionality of the application
logic.  While this way of structuring code is sometimes presented as a
virtue (``Event-driven programming''), it has an unfortunate side
effect, namely that event handlers are executed in a null context, so
that it becomes hard to even remember two consecutive events.  The
effect of event-driven programming is that applications written that way
have very rudimentary interaction policies.

At the lowest level, CLIM also has an event loop, but most application
programmers never have any reason to program at that level with CLIM.
Instead, CLIM has a \emph{command loop}
at a much higher level than the event loop.  At each iteration of the
command loop:

\begin{enumerate}
\item
 A command is acquired.  You might satisfy this demand by clicking on a
  menu item, by typing the name of a command, by hitting some kind of
  keystroke, by pressing a button, or by pressing some visible object
  with a command associated with it;
\item
 Arguments that are required by the command are acquired.  Each argument
    is often associated with a \emph{presentation type}, and visible
    objects of the right presentation type can be clicked on to satisfy
    this demand.  You can also type a textual representation of the
    argument, using completion, or you can use a context menu;
\item
 The command is called on the arguments, usually resulting in some
    significant modification of the data structure representing your
    application logic;
\item
 A \emph{display routine} is called to update the views of the
  application logic.  The display routine may use features such as
  incremental redisplay.
\end{enumerate}

Instead of attaching event handlers to gadgets, writing a CLIM
application therefore consists of:

\begin{itemize}
\item
 writing CLIM commands that modify the application data structures
  independently of how those commands are invoked, and which may take
  application objects as arguments;
\item
 writing display routines that turn the application data structures (and
  possibly some "view" object) into a collection of visible
  representations (having presentation types) of application objects;
\item
 writing completion routines that allow you to type in application
  objects (of a certain presentation type) using completions;
\item
 independently deciding how commands are to be invoked (menus, buttons,
  presentations, textual commands, etc).
\end{itemize}

By using CLIM as a mediator of command invocation and argument
acquisition, you can obtain some very modular code.  Application logic
is completely separate from interaction policies, and the two can evolve
separately and independently.
