\documentclass{sig-alternate-05-2015}
\usepackage[utf8]{inputenc}
\usepackage{color}

\def\inputfig#1{\input #1}
\def\inputtex#1{\input #1}
\def\inputal#1{\input #1}
\def\inputcode#1{\input #1}

\inputtex{logos.tex}
\inputtex{refmacros.tex}
\inputtex{other-macros.tex}

\begin{document}
\setcopyright{rightsretained}
\title{McCLIM Demonstration}
\numberofauthors{1}
\author{\alignauthor 
Daniel Kochmański\\
\affaddr{TurtleWare – Daniel Kochmański}\\
\affaddr{Przemyśl, Poland}\\
\email{daniel@turtleware.eu}}

\toappear{Permission to make digital or hard copies of all or part of
  this work for personal or classroom use is granted without fee
  provided that copies are not made or distributed for profit or
  commercial advantage and that copies bear this notice and the full
  citation on the first page. Copyrights for components of this work
  owned by others than the author(s) must be honored. Abstracting with
  credit is permitted. To copy otherwise, or republish, to post on
  servers or to redistribute to lists, requires prior specific
  permission and/or a fee. Request permissions from
  Permissions@acm.org.

  ELS '17, April 3 -- 6 2017, Brussels, Belgium
  Copyright is held by the owner/author(s).
}

\maketitle

\begin{abstract}
  We describe what is a \emph{Common Lisp Interface
    Manager}(CLIM)\cite{CLIM2} implementation called
  McCLIM\cite{Strandh:2002:ILC:McCLIM}. In particular, we describe
  recent improvements of the code base. We illustrate McCLIM and
  recent development by developing a demo application ``Clamber'',
  which is a book collection managament tool, which was created in
  purpose of explaining \emph{CLIM} concepts in form of a tutorial.
\end{abstract}

\begin{CCSXML}
  <ccs2012> <concept>
  <concept_id>10011007.10011006.10011066.10011069</concept_id>
  <concept_desc>Software and its engineering~Integrated and visual
  development environments</concept_desc>
  <concept_significance>500</concept_significance> </concept>
  </ccs2012>
\end{CCSXML}

\ccsdesc[500]{Software and its engineering~Integrated and visual
  development environments}

\printccsdesc

\keywords{\commonlisp{}, graphic user interfaces}

\section{Introduction}

The CLIM specification is large and requires some initial work from
the programmer to start writing programs using CLIM. Needless to say,
the implementation of such specification is a non-trivial undertaking
and indeed, McCLIM development consumed many man-hours of a bright
developers through the first decade of XXI century. In the second
decade development pace slowed down a little to get in track again
during the last few years, thanks to individuals like Robert Strandh
(current project leader), Alessandro Serra and others.

The \emph{CLIM 2.0} specification was released in 1993. It is meant to
be portable across Common Lisp\cite{ansi:common:lisp} implementations
and operating systems by mapping itself to the target window system
(like X11\cite{X11}, GTK\cite{GTK} or Windows
Forms\cite{WindowsForms}) through backends.

The key feature of the standard is providing a convenient
object-oriented abstraction over the interface presented to the user
(without compromising programmer's \emph{ability} to change low-level
details). With these things in mind it is a specification which
implements the model-view-presenter architectural pattern in a
consistent way while also providing defaults and the ability to
customize its behavior.

McCLIM is a free open source implementation of CLIM II specification
with extensions proposed by Franz Inc. in the \emph{CLIM 2 User Guide,
  version 2.2.2}. As of 2017, McCLIM (and recently opensourced
clim2\cite{clim2-franz}) is the only available native graphic user
interface toolkit available to the Common Lisp ecosystem . Other
solutions are based on foreign tools (LTK\cite{LTK},
CommonQt\cite{CommonQt} or EQL5\cite{EQL5}) or are commercial (Common
Graphics\cite{CG}, CAPI\cite{CAPI}). Another frequently used approach
is creating web applications with frameworks.

\section{Recent development}

During the last year we have replaced various functionalities with the
third party systems. The \emph{Bordeaux Threads}\cite{BT} library is
responsible for threading, \emph{OptiCL}\cite{OptiCL} for handling
raster images etc. We have also refreshed the McCLIM website to
provide up-to-date information. Some work has been done to assure its
portability across implementations. As of 2017-01-25 it is known to
work on \emph{CCL}\cite{CCL}, \emph{ECL}\cite{ECL} and
\emph{SBCL}\cite{SBCL}.

After that we have started a crowd funding campaign which allowed us
to finance bounties and some constant development work (40 hours per
month). It was a huge success which assured us, that people are lively
interested in seeing \emph{McCLIM} working. We have gathered very
valuable feedback which helped us to plan a roadmap\cite{roadmap} for
the development. We regularily present an iteration summary to the
community. We have noticed a development steady progress and growing
interest in using McCLIM by the Common Lisp developers.


A lot of effort is put in \emph{McCLIM} internals to assure quality
and ease of use, and to provide a learning material for new users. One
of the most discouraging parts of using CLIM is the software
complexity with a little of material which helps to understand how to
use it.
\section{Demonstration}

Key CLIM components are:

\begin{itemize}
\item Application Frames

  Application frame is a class abstracting one application entity. It
  is composed of the application internal state, its panes, layouts
  and commands associated with it.

\item Panes, layouts and views

  Panes are more widely known as widgets. Layout is a special kind of
  pane responsible for the application frame screen composition. Views
  may be used to customize the rendering according to the particular
  needs (textual view, gadget view, list view, grid view etc).

\item Drawing primitives and formatted output

  For custom output programmer is equipped with a rich set of drawing
  functions which may be used on a sheet. Also output may be arranged
  in tables, graphs and lists which helps to lay things on a sheet
  when writing custom widgets.

\item Presentation types

  Presentation of Lisp data displayed on the screen. This allows
  manipulating underlying Lisp objects with a pointer, supplying them
  to the commands and requesting input from the user.

\item Commands

  Commands are functions provided by the application to achieve its
  design goals. Command arguments may be gathered by various means
  available for the user – through a gesture, by selecting command in
  a menu or simply typed in the \emph{Listener}\cite{listener}, what
  allows a clear separation of application logic and its look.
\end{itemize}

\section{Conclusions and future work}

We have demonstrated that \emph{McCLIM} is an interesting piece of
software which allows creating customizable user
interfaces. Convenient abstraction is enables programmer to think in
high-level terms which are well mapped to the underlying
functionality.

Recent spawn of interest is a proof, that free CLIM implementation is
something demanded and we plan to improve it to the point, where it is
stable enough for production use. One of the primary goals is
providing a good documentation and learning materials for newcomers,
since the specification is vast (and not necessarily consistent).

\section{Acknowledgments} 

We would like to thank all the contributors, supporters and
programmers using \emph{McCLIM} for all the valuable feedback, code
improvement, financial support and encouraging words.

I personally want to thank professor Robert Strandh for his constant
patience and mentorship which made this work possible and
entertaining.

\bibliographystyle{abbrv}
\bibliography{mcclim-demo}
\end{document}
