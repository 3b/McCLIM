\chapter{Concepts}

\section{Coordinate systems}

CLIM uses a number of different coordinate systems and transformations
to transform coordinates between them.

The coordinate system used for the arguments of drawing functions is
called the \emph{user coordinate system},
and coordinate values expressed in the user coordinate system are known
as \emph{user coordinates}.

Each sheet has its own coordinate system called the \emph{sheet
coordinate system},
 and positions expressed in this coordinate system are said to be
expressed in \emph{sheet coordinates}.
  User coordinates are translated to \emph{sheet coordinates} by means
of the \emph{user transformation} also called the \emph{medium
transformation}.  This transformation is stored in the \emph{medium}
used for drawing.  The medium transformation can be composed temporarily
with a transformation given as an explicit argument to a drawing
function.  In that case, the user transformation is temporarily modified
for the duration of the drawing.

Before drawing can occur, coordinates in the sheet coordinate system
must be transformed to \emph{native coordinates}, which are coordinates
of the coordinate system of the native windowing system.  The
transformation responsible for computing native coordinates from sheet
coordinates is called the \emph{native transformation}.  Notice that
each sheet potentially has its own native coordinate system, so that the
native transformation is specific for each sheet.  Another way of
putting it is that each sheet has a mirror, which is a window in the
underlying windowing system.  If the sheet has its own mirror, it is the
\emph{direct mirror} of the sheet.  Otherwise its mirror is the direct
mirror of one of its ancestors.  In any case, the native transformation
of the sheet determines how sheet coordinates are to be translated to
the coordinates of that mirror, and the native coordinate system of the
sheet is that of its mirror.

The composition of the user transformation and the native transformation
is called the \emph{device transformation}.  It allows drawing
functions to transform coordinates only once before obtaining native
coordinates.

Sometimes, it is useful to express coordinates of a sheet in the
coordinate of its parent.  The transformation responsible for that is
called the \emph{sheet transformation}.

\section{Arguments to drawing functions}

Drawing functions are typically called with a sheet as an argument.

A sheet often, but not always, corresponds to a window in the underlying
windowing system.
