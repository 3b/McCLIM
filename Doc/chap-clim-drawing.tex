\chapter{CLIM drawing functions}

CLIM generalizes the concept of a hierarchy of window in a windowing
system in several different ways.  A window in a windowing system
generalizes to a \emph{sheet} in CLIM.  More precisely, a window in a
windowing system generalizes to the \emph{sheet region} of a sheet.  A
CLIM sheet is an abstract concept with an infinite \emph{drawing plane}
and the \emph{region} of the sheet is the potentially visible part of
that drawing plane.

CLIM \emph{sheet region}s don't have to be rectangular the way windows
in most windowing systems have to be.  Thus, the width and the height of
a window in a windowing system generalizes to an arbitrary
\emph{region} in CLIM.  A CLIM region is simply a set of mathematical
points in a plane.  CLIM allows this set to be described as a
combination (union, intersection, difference) of elementary regions made
up of rectangles, polygons and ellipses.

Even rectangular regions in CLIM are generalizations of the width+height
concept of windows in most windowing systems.  While the upper left
corner of a window in a typical windowing system has coordinates (0,0),
that is not necessarily the case of a CLIM region.  CLIM uses that
generalization to implement various ways of scrolling the contents of a
sheet.  To see that, imagine just a slight generalization of the
width+height concept of a windowing system into a rectangular region
with x+y+width+height.  Don't confuse the x and y here with the position
of a window within its parent, they are different.  Instead, imagine
that the rectangular region is a hole into the (infinite) drawing plane
defined by all possible coordinates that can be given to drawing
functions.  If graphical objects appear in the window with respect to
the origin of some coordinate system, and the upper-left corner of the
window has coordinates (x,y) in that coordinate system, then changing x
and y will have the effect of scrolling.

CLIM sheets also generalize windows in that a window typically has
pixels with integer-value coordinates.  CLIM sheets, on the other hand,
have infinte resolution.  Drawing functions accept non-integer
coordinate values which are only translated into integers just before
the physical rendering on the screen.

The x and y positions of a window in the coordinate system of its parent
window in a typical windowing system is a translation transformation
that takes coordinates in a window and transform them into coordinates
in the parent window.  CLIM generalizes this concepts to arbitrary
affine transformations (combinations of translations, rotations, and
scalings).  This generalization makes it possible for points in a sheet
to be not only translated compared to the parent sheet, but also rotated
and scaled (including negative scaling, giving mirror images).  A
typical use for scaling would be for a sheet to be a zoomed version of
its parent, or for a sheet to have its y-coordinate go the opposite
direction from that of its parent.

When the shapes of, and relationship between sheets are as simple as
those of a typical windowing system, each sheet typically has an
associated window in the underlying windowing system.  In that case,
drawing on a sheet translates in a relativly straightforward way into
drawing on the corresponding window.  CLIM sheets that have associated
windows in the underlying windowing system are called \emph{mirrored
sheets} and the system-dependent window object is called the
\emph{mirror}.  When shapes and relationships are more complicated,
CLIM uses its own transformations to transform coordinates from a sheet
to its parent and to its grandparent, etc., until a \emph{mirrored
sheet} is found.  To the user of CLIM, the net effect is to have a
windowing system with more general shapes of, and relationships between
windows.
