% -*- coding: utf-8; -*-
\documentclass{article}

\usepackage[T1]{fontenc}
\usepackage[utf8x]{inputenc}
\usepackage{color}
\usepackage{epsfig}
\usepackage{alltt}
\usepackage{moreverb}

\ifx\pdfoutput\undefined \csname newcount\endcsname\pdfoutput \fi 
\ifcase\pdfoutput  \else 
\usepackage[pdftex]{hyperref}
\fi

\setlength{\parskip}{0.3cm}
\setlength{\parindent}{0cm}

\def\inputfig#1{\input #1}

\newenvironment{itemize0}{
\begin{itemize}
\setlength{\parskip}{0cm}%
}
{\end{itemize}}

\newenvironment{enumerate0}{
\begin{enumerate}
\setlength{\parskip}{0cm}%
}
{\end{enumerate}}

\input spec-macros

\newcommand{\gloss}[1]{\textsl{\textcolor{red}{#1}}}
\newcommand{\glossentry}[1]{\paragraph{#1}}
\newcommand{\class}[1]{\texttt{#1}}
\newcommand{\genfun}[1]{\texttt{#1}}
\newcommand{\macro}[1]{\texttt{#1}}
\newcommand{\gadget}[1]{\texttt{#1}}
\newcommand{\pane}[1]{\texttt{#1}}
\newcommand{\initarg}[1]{\texttt{#1}}
\newcommand{\methcomp}[1]{\texttt{#1}}
\newcommand{\slot}[1]{\texttt{#1}}
\newcommand{\code}[1]{\texttt{#1}}

\newcommand{\longref}[1]{(See \ref{#1})}
\newcommand{\var}[1]{\textit{#1}}

\newcommand{\nil}[0]{\cl{nil}}
\title{CLIM Sheet Hierarchies}
\author{Robert Strandh \\ strandh@labri.fr}

\begin{document}

\maketitle

{\setlength{\parskip}{0cm}
\tableofcontents}

\section{Sheet hierarchies}

CLIM sheets are organized into a hierarchy.  Each sheet has a sheet
transformation and a sheet region.  The sheet tranformation determines
how coordinates in the sheet's own coordinate system get translated
into coordinates in the coordinate system of its parent.  The sheet
region determines the \gloss{potentially visible area} of the
otherwise infinite drawing plane of the sheet.  The sheet region is
given in the coordinate system of the sheet. 

In McCLIM, every grafted sheet has a \gloss{native transformation}.
The native transformation is used by drawing functions to translate
sheet coordinates to \gloss{native coordinates}, so that drawing can
occur on the (not necessarily immediate) mirror of the sheet.  It
would therefore be enough for sheets that support the \gloss{output
protocol} to have a native transformation.  However, it is easier to
generalize it to all sheets, in order to simplify the programming of
the computation of the native transformation.  Thus, in McCLIM, even
sheets that are mute for output have a native transformation.

In McCLIM, every grafted sheet also has a \gloss{native region}.  The
native region is intersection the sheet region and the region of all
of its ancestors, except that the native region is given in
\gloss{native coordinates}, i.e. the coordinates obtained after the
application of the \gloss{native transformation} of the sheet.

\section{Computing the native transformation}

CLIM maintains, for each sheet that supports output, a \gloss{native
transformation}.  The native transformation is used by drawing
functions to transform coordinates to \gloss{native coordinates},
i.e., those required by the windowing system.  In this section, we
examine how that computation is made.

To get an idea of the complexity involved, let us imagine a hierarchy
of three sheets: A, B, and C as shown in figure \ref{fignative}.  The
name of each sheet is placed close to the origin of its own coordinate
system.  The region of B is marked with rb and is some arbitrary
shape.  The region of C is marked rc and has the shape of an ellipse.
We do not need the region of A for our example.  In the example, A and
C are mirrored, but B is not.  The origin of the mirror of A is marked
with a lower-case a and the origin of C is marked with a lower-case
c.  Mirrors are outlined with dotted rectangles. 

\begin{figure}
\begin{center}
\inputfig{native.pstex_t}
\end{center}
\caption{\label{fignative} A sheet with a nontrivial transformation}
\end{figure}

In our example, the sheet transformations are nontrivial.  Notice that
the x-axis in the coordinate system of C grows downward and that the
y-axis in the coordinate system of B grows upward.  

As defined in the specification, a transformation hffas six components:
$m_{xx}$, $m_{xy}$, $m_{yx}$, $m_{yy}$, $t_x$, and $t_y$ such that 

$$x' = m_{xx} x + m_{xy} y + t_x$$ and $$y' = m_{yx} x + m_{yy} y + t_y$$

Let us first establish the sheet transformations of B and C in the
example.  By transforming the point (0,0) of B to the coordinate
system of A we get:

$$45 = m_{xx} \cdot 0 + m_{xy} \cdot 0 + t_x$$
$$5 = m_{yx} \cdot 0 + m_{yy} \cdot 0 + t_y$$

so $t_x = 45$ and $t_y = 5$. For the point (0,10) we then get:

$$40 = m_{xx} \cdot 0 + m_{xy} \cdot 10 + 45$$
$$5 = m_{yx} \cdot 0 + m_{yy} \cdot 10 + 5$$

which gives $m_{xy} = -0.5$ and $m_{yy} = 0$.  Finally, for the point
(10,0) we get:

$$45 = m_{xx} \cdot 10 - 0.5  \cdot 0 + 45$$
$$15 = m_{yx} \cdot 10 + 0 \cdot 0 + 5$$

which gives $m_{xx} = 0$ and $m_{yx} = 1$.

Using the notation:

\[ \left[ \begin{array}{ccc}
m_{xx} & m_{xy} & t_x\\
m_{yx} & m_{yy} & t_y 
\end{array} \right] \]

we get 

\[ T_B = \left[ \begin{array}{ccc}
0 & -0.5 & 45\\
1 & 0 & 5 
\end{array} \right] \]

where $T_s$ means the sheet transformation of the sheet $s$. 

Let us now attack the sheet transformation of C.  For the point (0,0)
in the coordinate system of C we get (30,70) in the coordinate system
of B, giving $t_x = 30$ and $t_y = 70$.  Transforming (0,10) gives
(25,70) so that:

$$25 = m_{xx} \cdot 0 + m_{xy} \cdot 10 + 30$$

and 

$$70 = m_{yx} \cdot 0 + m_{yy} \cdot 10 + 70$$

Giving $m_{xy} = -0.5$ and $m_{yy} = 0$
Finally, transforming (10,0) gives (30,60) so that

$$30 = m_{xx} \cdot 10 + m_{xy} \cdot 0 + 30$$

and

$$60 = m_{yx} \cdot 10 + m_{yy} \cdot 0 + 70$$

which gives $m_{xx} = 0$ and $m_{yx} = -1$

Thus

\[ T_C = \left[ \begin{array}{ccc}
0 & -0.5 & 25\\
-1 & 0 & 70 
\end{array} \right] \]

Let us now attack the native transformations (written $N_s$).  For
that, we need to know the native transformation of A, i.e., $N_A$.
Let us for the moment assume that the native transformation of A is a
simple translation with $t_x = 0$ and $t_y = -10$.  This implies that
the \emph{scale} of the coordinate system of A is directly in pixels,
so that the mirror c is about 15 pixels tall and 10 pixels wide.

Thus we have that 

\[ N_A = \left[ \begin{array}{ccc}
1 & 0 & 0\\
0 & 1 & -10 
\end{array} \right] \]

To compute the native transformation of B, we observe that drawing on
B must end up on the mirror a of A.  Let us first see whether we can
figure it out manually the same way we did for the sheet
transformations.  We get the system of equations:

\[ \begin{array}{ccccccc}
45 & = & m_{xx} \cdot 0 & + & m_{xy} \cdot 0 & + & t_x \\
-5 & = & m_{yx} \cdot 0 & + & m_{yy} \cdot 0 & + & t_y \\
40 & = & m_{xx} \cdot 0 & + & m_{xy} \cdot 10 & + & t_x \\
-5 & = & m_{yx} \cdot 0 & + & m_{yy} \cdot 10 & + & t_y \\
45 & = & m_{xx} \cdot 10 & + & m_{xy} \cdot 0 & + & t_x \\
5 & = & m_{yx} \cdot 10 & + & m_{yy} \cdot 0 & + & t_y \\
\end{array} \]

which gives

\[ N_B = \left[ \begin{array}{ccc}
0 & -0.5 & 45\\
1 & 0 & -5 
\end{array} \right] \]

Similarly, to compute $N_C$, we solve the system:

\[ \begin{array}{ccccccc}
10 & = & m_{xx} \cdot 0 & + & m_{xy} \cdot 0 & + & t_x \\
25 & = & m_{yx} \cdot 0 & + & m_{yy} \cdot 0 & + & t_y \\
10 & = & m_{xx} \cdot 0 & + & m_{xy} \cdot 10 & + & t_x \\
20 & = & m_{yx} \cdot 0 & + & m_{yy} \cdot 10 & + & t_y \\
15 & = & m_{xx} \cdot 10 & + & m_{xy} \cdot 0 & + & t_x \\
25 & = & m_{yx} \cdot 10 & + & m_{yy} \cdot 0 & + & t_y \\
\end{array} \]

Which gives

\[ N_C = \left[ \begin{array}{ccc}
0.5 & 0 & 10\\
0 & -0.5 & 25 
\end{array} \right] \]

Now, let us try to find a way to compute $N_B$ from $N_A$ and $T_B$.
Drawing on B must end up on the mirror a of A, so it seems reasonable
to think that you should first apply $T_B$ and then $N_A$ to the
control points of the figure drawn.  But this is the composition of
$N_A$ and $T_B$, written $N_A \circ T_B$.  The composition of two
transformations $m' \circ m$ is computed as follows:

\[ \left[ \begin{array}{ccc}
{m'}_{xx} m_{xx} + {m'}_{xy} m_{yx} & {m'}_{xx} m_{xy} + {m'}_{xy}
m_{yy} & {m'}_{xx} t_x + {m'}_{xy} t_y + {t'}_x\\
{m'}_{yx} m_{xx} + {m'}_{yy} m_{yx} & {m'}_{yx} m_{xy} + {m'}_{yy}
m_{yy} & {m'}_{yx} t_x + {m'}_{yy} t_y + {t'}_y
\end{array} \right] \]

Using that formula, we get

\[ N_A \circ T_B = \left[ \begin{array}{ccc}
1 \cdot 0 + 0 \cdot 1 & 1 \cdot -0.5 + 0
\cdot 0 & 1 \cdot 45 + 0 \cdot 5 + 0\\
0 \cdot 0 + 1 \cdot 1 & 0 \cdot -0.5 + 1
\cdot 0 & 0 \cdot 45 + 1 \cdot 5 - 10
\end{array} \right] \]

which gives

\[ N_A \circ T_B = \left[ \begin{array}{ccc}
0 & -0.5 & 45\\
1 & 0 & -5
\end{array} \right] \]

which corresponds exactly to our manual calculation of $N_B$ above. 

Computing the native transformation of C is more complicated, since C
is mirrored.  First, let us introduce the notion of a \gloss{mirror
transformation}.  The mirror transformation is similar to the sheet
transformation, in that it transforms coordinates of a mirror to
coordinates of its parent mirror.  We use $M_x$ to denote the mirror
transformation of the mirror of the sheet $x$.  The mirror
transformation is obviously port specific.  With most windowing
systems (and certainly with X) this transformation will be a simple
translation. 

Drawing on C should have the same effect as drawing on B with $T_C$
applied to the control points, and that should have the same effect as
drawing on the mirror of B with first $N_C$ and then $M_C$ applied 
to the control points.  In other words $N_B \circ T_C = M_C \circ
N_C$.  Applying the inverse of $M_C$ to both sides gives $M_C^{-1}
\circ N_B \circ T_C = M_C^{-1} \circ M_C \circ N_C = N_C$ 

We now have a way of computing $NC$, namely the inverse of the mirror
transformation of C composed with the native transformation of B
composed with the sheet transformation of C. 

Now we have an algorithm for computing native transformations.  To
simplify that algorithm, let us define the native transformation for a
\gloss{mute} sheet as well (it will have the same definition as that
for a non-mirrored sheet). 

\begin{enumerate0}
  
\item The native transformation of the top level sheet is decided by
  the frame manager.  It may be the \gloss{identity transformation} or
  a transformation from millimeters to pixels, or whatever the frame
  manager decides.
  
\item The native transformation of any mirrored sheet other than the
  top level sheet is the composition of the inverse of its mirror
  transformation, the native transformation of its parent, and its own
  sheet transformation.
  
\item The native transformation of any sheet without a mirror is the
  composition of the native transformation of its parent and its own
  sheet transformation.
\end{enumerate0}

\section{Computing the native region}

Given the definition above of native region, we must now find an
algorithm for computing it.  

For a sheet without a mirror, the native coordinate system is the same
as that of its parent.  Thus, the native region of such a sheet is the
sheet transformation of the sheet transformed by the native
transformation of the sheet and then intersected with the native
region of its parent. 

For a mirrored sheet $S$, things are slightly more complicated,
because the native coordinate system of such a sheet and of its parent
are not the same.  They differ by an application of the \gloss{mirror
transformation}.  The native region of its parent is expressed in the
native coordinate system of its parent, whereas transforming the sheet
region of $S$ by the sheet transformation $T_S$ of $S$ yields a region
in the native coordinate system of $S$.  We need to intersect that
region with the native region of its parent expressed in the native
coordinate system of $S$.  We must therefore apply the inverse of the
mirror transformation $M_S$ of $S$ to the native region of the parent
of $S$ before intersecting the two.   We get the following algorithm,
which assumes that the native transformation of $S$ has already been
computed: 

\begin{enumerate0}
\item if $S$ is not mirrored, compute its native region as its sheet
region transformed by its native transformation and then intersected
with the native region of its parent.
\item if $S$ is mirrored, compute its native region as its sheet
region transformed by its native transformation and then intersected
with the native region of its parent transformed by the inverse of the
mirror transformation of $S$.
\end{enumerate0}

\section{Moving and resizing sheets and regions}
\label{secmoving}

In this section, we discuss moving and resizing.  In particular, we
determine the sharing of responsibilities for moving and resizing
sheets and regions between a sheet and its parent. 

A common operation necessary in CLIM is to move a sheet or
its region.  There are many variants of moving a sheet and its
region:

\begin{itemize0}
\item The origin of the coordinate system of a sheet may be moved with
respect to that of its parent.  The position of the \gloss{sheet
region} then also changes in the coordinate system of the parent of
the sheet;
\item The region of a sheet may be moved with respect to the origin of
the coordinate system of that sheet.  The position of the \gloss{sheet
region} then also changes in the coordinate system of the parent of
the sheet;
\item The two previous operations can occur simultaneously with
opposite moves.  This gives the effect of scrolling, i.e. the region
of the sheet remains in the same place in the coordinate system of the
parent of the sheet.
\end{itemize0}

Let us first examine the responsibilities of a \gloss{layout pane}.
Such a pane is in charge of moving and resizing regions of its
children.  Several questions must be answered:

\begin{itemize0}
\item When moving a region of a child, should the layout pane move
the sheet transformation of its child, the sheet region of its child
within the coordinate system of its child, or a combination of the
two?
\item When resizing a region of a child, how should the layout pane
align the new region with respect to the old one (this would influence
what part of the old region is still visible in the resized one)?
\end{itemize0}

To answer the first question, we simply observe that we would like the
visible contents of the child to remain unchanged whenever its region
is moved by the layout pane.  This behavior implies that the position
of the region of the child in the coordinate system of that child
remain unchanged.  Thus, when a layout pane must move the region of
its child in its own coordinate system, it must do so by only moving
the \gloss{sheet transformation} of the child.

Conversely, if a sheet needs to move its region with respect to its
own coordinate system, for instance to obtain the effect of scrolling,
it must make sure that its region does not move in the coordinate
system of its parent.  Every such move must therefore be compensated
by a simultaneous opposite move of the \gloss{sheet transformation}.
Moving only the position of the region without changing the sheet
transformation would defeat the work of a potential \gloss{layout
pane} parent. 

To answer the second question, we imagine several different types of
children.  An ordinary stream pane would probably like to align the
lower-left corner of the old and the new region so that the first
column of the last line of text is still visible.  A stream pane for
arabic or hebrew would prefer to align the lower-right corner.  And a
stream for an asian language written from top to bottom, right to
left, would like to preserve the upper-left corner.  In other words,
the layout pane cannot make this decision, as it depends on the exact
type of the child.  The layout pane must therefore delegate this
decision to the child.  

\section{Scrolling}

Scrolling generally means moving the origin of the coordinate system
of a sheet with respect to the visible part of its region, thus making
some different part of the drawing plane of the sheet visible. 

There are two types of scrolling, conceptually completely different:

\begin{itemize0}
\item the scrolling of a pane not contained in (the viewport of) a
scroller pane, 
\item the scrolling of a pane contained in (the viewport of) a
scroller pane. 
\end{itemize0}

In particular, we need to know how a stream pane should behave when a
new line of text needs to be displayed, and that new line is outside
the current region of the stream pane.

When a pane is \emph{not} contained in a scroller pane, scrolling must
preserve the size of its region and the position of its region in the
coordinate system of its parent (see section \ref{secmoving}).  The
normal scrolling action for a stream pane not contained in a scroller
pane would therefore be to translate its region in the positive y
direction by an amount that corresponds to the hight of a line, and to
translate itself (i.e. to alter its sheet transformation) in the
negative y direction by the same amount.

When a pane \emph{is} contained in a scroller pane, the situation is
completely different.  The viewport of the scroller pane does not
behave like an ordinary layout pane (see section \ref{secmoving}) in
that it does not have an opinion about the size and position of the
region of its child.  The scroller pane \emph{itself} is a layout pane
that controls the size and position of the region of the
\gloss{viewport pane} which is one of its children (the others are the
scroll bars), but the viewport pane does not control the size and
position of the region of \emph{its} child.  Instead, the scroller
pane simply alters the appearance of the scroll bars to reflect the
size and position of region of the child of the viewport with respect
to the size and position of the region of the viewport itself.

A stream pane contained in a scroller pane should therefore behave
differently from a stream pane not contained in a scroller pane in
order for the scroller pane to serve any purpose.  But exactly how
should it behave?  First, with respect to the size of its region, its
normal scrolling action should probably make its region bigger (at
least up to a certain point) so that its previous contents could be
viewed by moving the scroll bars.  

But what about the \emph{position} of the region?  This question is in
fact two different questions:

\begin{itemize0}
\item How should the position of the region in the coordinate system of the
stream pane be altered (if at all)?
\item How should the position of the region in the coordinate system of the
viewport pane be altered (if at all)?
\end{itemize0}

To answer the first question, the most reasonable thing to do when the
region is made bigger would probably be to maintain the upper edge of
the region and extend the region downwards (or rather, in the
direction of the positive y axis).  When the region is not made bigger
(perhaps because some limit has been set), the region should simply
be translated in the direction of positive y axis. 

The second question is harder, and the answer is ``it depends''.  Let
us study the behavior of something similar, namely the X11 application
\texttt{xterm}.  Whenever any output whatsoever is made in the window
of an \texttt{xterm} with scroll bars, the line where the output takes
place is put on display.  To see that, start an \texttt{xterm} with
scroll bars; make sure the scroll bar is smaller than the size of the
window; scroll up; then type a character.  To obtain this behavior for
a CLIM stream pane, any output to the stream pane should align the
bottom of the stream pane region with the bottom of the viewport
region.  But this must be done by moving only the sheet transformation
of the stream pane.  

Another possibility is, of course, would be to always maintain the
position of the bottom of the region of the stream pane with respect
to the bottom of the region of the viewport pane.  That way, special
action would only need to be taken by the stream pane as a result of
the desire to scroll, and not as a result of \emph{any} output.  New
lines of output might not become immediately visible if the scroll bar
is not in its bottom position.  To obtain this effect, the stream pane
could simply translate its sheet transformation in the negative y
direction to compensate for the increased size.  This solution is
probably the simplest one. 

\end{document}