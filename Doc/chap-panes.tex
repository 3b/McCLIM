\chapter{Panes}
\index{pane}

Panes are subclasses of sheets.  Some panes are \emph{layout panes}
that determine the size and position of its children according to rules
specific to each particular type of layout pane.  Examples of layout
panes are vertical and horizontal boxes, tables etc.

According to the CLIM specification, all CLIM panes are
\emph{rectangular objects}.  For McCLIM, we interpret that phrase to
mean that:

\begin{itemize}
\item
 CLIM panes appear rectangular in the native windowing system;
\item
 CLIM panes have a native transformation that does not have a rotation
  component, only translation and scaling.
\end{itemize}

Of course, the specification is unclear here.  Panes are subclasses of
sheets, and sheets don't have a shape per-se.  Their \emph{regions} may
have a shape, but the sheet itself certainly does not.

The phrase in the specification \emph{could} mean that the
\emph{sheet-region} of a pane is a subclass of the region class
\emph{rectangle}.  But that would not exclude the possibility that the
region of a pane would be some non-rectangular shape in the
\emph{native coordinate system}.  For that to happen, it would be
enough that the \emph{sheet-transformation} of some ancestor of the
pane contain a rotation component.  In that case, the layout protocol
would be insufficient in its current version.

McCLIM panes have the following additional restrictions:

\begin{itemize}
\item
 McCLIM panes have a coordinate system that is only a translation
  compared to that of the frame manager;
\item
 The parent of a pane is either nil or another pane.
\end{itemize}

Thus, the panes form a \emph{prefix} in the hierarchy of sheets.  It is
an error for a non-pane to adopt a pane.

Notice that the native transformation of a pane need not be the identity
transformation.  If the pane is not mirrored, then its native
transformation is probably a translation of that of its parent.

Notice also that the native transformation of a pane need not be the
composition of the identity transformation and a translation.  That
would be the case only of the native transformation of the top level
sheet is the identity transformation, but that need not be the case.  It
is possible for the frame manager to impose a coordinate system in (say)
millimeters as opposed to pixels.  The native transformation of the top
level sheet of such a frame manager is a scaling with coefficients other
than 1.

\section{Layout protocol}
\index{layout protocol}

There is a set of fundamental rules of CLIM dividing responsibility
between a parent pane and a child pane, with respect to the size and
position of the region of the child and the \emph{sheet transformation}
of the child.  This set of rules is called the \emph{layout protocol}.

The layout protocol is executed in two phases.  The first phase is
called the \emph{space compostion} phase, and the second phase is
called the \emph{space allocation} phase.

\subsection{Space composition}

The space composition is accomplished by the generic function
\texttt{compose-space}.  When applied to a pane, \texttt{compose-space}
returns an object of type \emph{space-requirement} indicating the needs
of the pane in terms of preferred size, minimum size and maximum size.
The phase starts when compose-space is applied to the top-level pane of
the application frame.  That pane in turn may ask its children for their
space requirements, and so on until the leaves are reached.  When the
top-level pane has computed its space requirments, it asks the system
for that much space.  A conforming window manager should respect the
request (space wanted, min space, max space) and allocate a top-level
window of an acceptable size.  The space given by the system must then
be distributed among the panes in the hierarchy
\refSec{space-allocation}.

Each type of pane is responsible for a different method on
\texttt{compose-space}.  Leaf panes such as \emph{labelled gadgets} may
compute space requirements based on the size and the text-style of the
label.  Other panes such as the vbox layout pane compute the space as a
combination of the space requirements of their children.  The result of
such a query (in the form of a space-requirement object) is stored in
the pane for later use, and is only changed as a result of a call to
\texttt{note-space-requirement-changed}.

Most \emph{composite panes} can be given explicit values for the values
of \texttt{:width}, \texttt{:min-width}, \texttt{:max-width},
\texttt{:height}, \texttt{:min-height}, and \texttt{:max-height}
options.  If such arguments are not given (effectively making these
values nil), a general method is used, such as computing from children
or, for leaf panes with no such reasonable default rule, a fixed value
is given.  If such arguments are given, their values are used instead.
Notice that one of \texttt{:height} and \texttt{:width} might be
given, applying the rule only in one of the dimensions.

Subsequent calls to \texttt{compose-space} with the same arguments are
assumed to return the same space-requirement object, unless a call to
note-space-requirement-changed has been called in between.

\subsection{Space allocation}
\label{space-allocation}

When \texttt{allocate-space} is called on a pane \texttt{P}, it must
compare the space-requirement of the children of \texttt{P} to the
available space, in order to distribute it in the most preferable way.
In order to avoid a second recursive invokation of
\texttt{compose-space} at this point, we store the result of the
previous call to \texttt{compose-space} in each pane.

To handle this situtation and also explicitly given size options, we use
an \texttt{:around} method on \texttt{compose-space}.  The
\texttt{:around} method will call the primary method only if necessary
(i.e., \texttt{(eq (slot-value pane 'space-requirement) nil)}), and store
the result of the call to the primary method in the
\texttt{space-requirement} slot.

We then compute the space requirement of the pane as follows:

\begin{verbatim}
     (setf (space-requirement-width ...)  (or explicit-width
           (space-requirement-width request)) ...
           (space-requirement-max-width ...)  (or explicit-max-width
           explicit-width (space-requirement-max-width request)) ...)
\end{verbatim}

When the call to the primary method is not necessary we simply return
the stored value.

The \texttt{spacer-pane} is an exception to the rule indicated above.  The
explicit size you can give for this pane should represent the margin
size.  So its primary method should only call compose on the child.  And
the around method will compute the explicit sizes for it from the space
requirement of the child and for the values given for the surrounding
space.

\subsection{Change-space Notification Protocol}

The purpose of the change-space notification protocol is to force a
recalculation of the space occupied by potentially each pane in the
\emph{pane hierarchy}.  The protocol is triggerred by a call to
\texttt{note-space-requirement-changed} on a pane \texttt{P}. In McCLIM, we
must therefore invalidate the stored space-requirement value and
re-invoke \texttt{compose-space} on \texttt{P}.  Finally, the
\emph{parent} of \texttt{P} must be notified recursively.

This process would be repeated for all the panes on a path from \texttt{P}
to the top-level pane, if it weren't for the fact that some panes
compute their space requirements independently of those of their
children.  Thus, we stop calling \texttt{note-space-requirement-changed}
in the following cases:

\begin{itemize}
\item
 when \texttt{P} is a \texttt{restraining-pane},
\item
 when \texttt{P} is a \texttt{top-level-sheet-pane}, or
\item
 when \texttt{P} has been given explicit values for \texttt{:width} and
  \texttt{:height}
\end{itemize}

In either of those cases, \texttt{allocate-space} is called.
