\documentclass[twocolumn,a4paper]{article}
\usepackage[dvips]{graphicx}
\usepackage{color}		% Need the color package
\usepackage{listings}
\usepackage{url}
%\usepackage{epsfig}
\title{\Huge A Guided Tour of CLIM, \\ Common Lisp Interface Manager}
\author{
2006 Update \\ 
Clemens Fruhwirth \texttt{<clemens@endorphin.org>} \\ The McCLIM Project 
\bigskip \\
\begin{tabular}{ll}
\multicolumn{2}{c}{Original article\footnote{Published in Lisp Pointers 1991}} \\ 
Ramana Rao \texttt{<rao@xerox.com>} &  Xerox Palo Alto Research Center \\
William M. York \texttt{<york@ila.com>} & International Lisp Associates, Inc. \\
Dennis Doughty \texttt{<doughty@ila.com>} &  International Lisp Associates, Inc. 
\end{tabular}
}

\def\VRfont{\sffamily}

\lstset{%
   basicstyle=\small\sffamily,
   keywordstyle=\small\sffamily,
   stringstyle=\rmfamily,
   commentstyle=\rmfamily\itshape,
   frame=lines,
   language=Lisp }

\lstdefinestyle{framestyle}
  {frame=lines}
\lstdefinestyle{inlinestyle}
  {frame=none}


\makeatletter
\newcommand {\concept} [1] {{\sl #1}\index{#1}}
\newcommand {\term} [1] {{\sl #1}}
\newcommand {\code}[1]{{\sffamily #1}}
\newcommand {\CLIM}{\textsc{clim}}
\newcommand {\CLOS}{\textsc{clos}}
\newcommand {\mcclim}{\textsc{McCLIM}}
\let\class\code
\let\method\code
\let\constant\code
\let\variable\code
\let\macro\code
\let\keyword\code
\makeatother

\begin{document}
\maketitle
\begin{abstract}
\noindent The Common Lisp Interface Manager (\CLIM{}) provides a
layered set of facilities for building user interfaces. These
facilities include a portable layers for basic windowing, input,
output services, and mechanisms for constructing window types and user
interface components; stream-oriented input and output facilities
extended with presentations and context sensitive
input;\footnote{Similar to the work pioneered in the Genera UI system}
and a gadget-oriented toolkit similar to those found in the X world
extended with support for look and feel adaptiveness. In this article,
we present an overview of \CLIM{}'s broad range of functionality and
present a series of examples that illustrates \CLIM{}'s power. The
article originally appeared in Lisp Pointers in 1991 and was updated
in 2006 by Clemens Fruhwirth.\footnote{The CLIM 2 specification
  changed significant parts of \CLIM{} rendering legacy code
  unusable. Clemens Fruhwirth has rewritten all examples and the 
  corresponding text sections for the \CLIM{} 2 specification. In
  addition, he has restructured the whole article, adding sections
  to provide additional insights into \CLIM{} concepts.} All examples
in this article have been run with \mcclim{}\cite{mcclim}, a free
\CLIM{} implementation, as of January 2006.
\end{abstract}

\section*{Introduction}
Common Lisp is a language standard that has provided a broad range of
functionality, and that has, to a large degree, successfully enabled
the writing of truly portable Lisp programs. The emergence of \CLOS{} and
the cleanup efforts of ANSI X3J13 have further enhanced the utility
and portability of Common Lisp. However, one major stumbling block
remains in the path of those endeavoring to write large portable
applications. The Common Lisp community has not yet provided a
standard interface for implementing user interfaces beyond the most
basic operations based on stream reading and printing.\footnote{Notice
  that this sentence was written in 1991; it is still
  true 15 years later.}

The Common Lisp Interface Manager addresses this problem by specifying
an interface to a broad range of services necessary or useful for
developing graphical user interfaces. These services include low level
facilities such as geometry, graphics, event-oriented input, and
windowing; intermediate level facilities such as support for Common Lisp
stream operations, output recording, and advanced output formatting;
and high level facilities such as context sensitive input, an adaptive
toolkit, and an application building framework.

\CLIM{} implementations will eventually support a large number of window environments
including X Windows, Mac OS X and Microsoft Windows. \CLIM{} is
designed to exploit the functionality provided by the host environment
to the degree that it makes sense. For example, \CLIM{} top level
windows are typically mapped onto host windows, and input and output
operations are ultimately performed by host window system
code. \CLIM{} supports the incorporation of
toolkits written in other languages. A uniform interface provided by
\CLIM{} allows Lisp application programmers to deal only with Lisp
objects and functions regardless of the operating platform.

An important goal that has guided the design of \CLIM{} was to
layer the specification into a number of distinct
facilities. Furthermore, the specification does not distinguish the
use of a facility by higher level \CLIM{} facilities from its use by
\CLIM{} users. For example, the geometry substrate, which includes
transformations and regions, is designed for efficient use by the
graphics and windowing substrates as well as by \CLIM{} programmers. This
means that, in general, a \CLIM{} programmer can reimplement higher level
\CLIM{} facilities using the interfaces provided by lower level
facilities.

This modular, layered design has a number of benefits. The \CLIM{}
architecture balances the goal of ease of use on one hand, and the
goal of versatility on the other. High level facilities allow
programmers to build portable user interfaces quickly, whereas lower
level facilities provide a useful platform for building toolkits or
frameworks that better support the specific needs or requirements of a
particular application.

For example, \CLIM{}'s application framework and adaptive toolkit
allow programmers to develop applications that automatically adopt the look
and feel of the host's environment. We often call this
``adaptiveness,'' ``look and feel independence,'' or occasionally more
picturesquely, ``chameleon look and feel''. However, many users may
need or want to define a particular look and feel that is constant
across all host environments (we call this ``portable look and
feel''). Such users can circumvent the look and feel adaptiveness
provided by \CLIM{}, while still using most of the application
framework facility and other high level \CLIM{} facilities like
context sensitive input. Furthermore, using the lower level facilities
of \CLIM{}, they can develop portable toolkit libraries that define
and implement their own particular look and feel. For instance, the
\CLIM{} programmer can implement new gadget types on top of the drawing
primitives and treat them equally to the built-in gadget types.

We will use the term \concept{CLIM implementor} for the party
implementing low-level and high-level parts according to the \CLIM{}
specification, \concept{CLIM programmer} for the party that will use
the facilities provided by the implementor, and \concept{CLIM user}
for the party that will use the programs provided by the programmer.

The next section presents an overview of the functionality provided by
\CLIM{} facilities.

\section{Overview of Functionality}

Figure \ref{clim-facilities} shows the various aspects of a host
environment in which \CLIM{} lives as well as the various elements
provided by \CLIM{}. Below we briefly describe a number of \CLIM{}'s
areas of functionality. Later sections will illustrate many of these
components.

\paragraph*{Geometry} \CLIM{} provides points, rectangles, and
transformations; and functions for manipulating these object types.

\begin{figure*}
\begin{center}
\input{techno-dep.pstex_t}
\end{center}
\caption{An Overview of \CLIM{} facilities}\label{clim-facilities}
\end{figure*}

\paragraph*{Graphics substrate} \CLIM{} provides a portable interface
to a broad set of graphics functions for drawing complex geometric
shapes.

\paragraph*{Windowing substrate} \CLIM{} provides a portable layer for
implementing sheets (windows and other window-like objects).

\paragraph*{Extended Streams} \CLIM{} integrates the Common Lisp
Stream I/O functionality with the \CLIM{} graphics, windowing, and
panes facilities. 
% I believe that this was what was intended [2006/03/14:rpg]
In addition to ordinary text, the programmer can send a
button, a picture or any other arbitrary widget to a \CLIM{} output
stream and \CLIM{} will display the widget in the sheet associated
with the output stream.

\paragraph*{Output Recording} \CLIM{} provides output recording for
capturing all output done to an extended stream and automatically
repainting it when necessary.

\paragraph*{Formatted Output} \CLIM{} provides a set of high-level
macros that enable programs to produce neatly formatted tabular and
graphical displays easily.\footnote{This also includes Graph
  Formatting.  Graph formatting is only partly implemented in McCLIM
  at this date (March 2006).}

\paragraph*{Presentations} \CLIM{} provides the ability to associate
semantics with output, such that Lisp objects may be retrieved later
via user gestures (e.g.{} mouse clicks) on their displayed
representation. This context sensitive input is modularly layered on
top of the output recording facility and is integrated with the Common
Lisp type system. 
% I understand this, but I suspect it's not going to be obvious to the
% ordinary reader why type coercion provides the basis for a user
% interface... [2006/03/14:rpg]
A mechanism for type coercion is also included,
providing the basis for powerful user interfaces.

\paragraph*{Panes} \CLIM{} provides \concept{panes} that are analogous
to the gadgets or widgets of toolkits like the X toolkit, GTK or Mac
OS X's toolkit.

Supported pane types include layout panes for arranging other panes,
gadget panes for presenting users with feedback information or
mechanisms for invoking application behavior, and application panes
for displaying and allowing users to interact with application data.

\paragraph*{Look and Feel Adaptiveness} \CLIM{} supports look and feel
independence by specifying a set of abstract gadget pane
protocols. These protocols define a gadget in terms of its function
and not in terms of the details of its appearance or
operation. Applications that use these gadget types and related
facilities will automatically adapt to use whatever toolkit is
available on and appropriate for the host environment. In addition,
portable Lisp-based implementations of the abstract gadget pane
protocols are provided.\footnote{\mcclim{} does not support look and
  feel adaptiveness at the moment except for the experimental beagle
  backend for Mac OS X's Cocoa platform, or the Gtkairo backend using
  GTK. Hence, \mcclim{} mostly uses this portable Lisp-based
  implementation.}

\paragraph*{Application Building} \CLIM{} provides a set of tools for
defining application frames. These tools allow the programmer to
specify all aspects of an application's user interface, including pane
layout, interaction style, look and feel, and command menus and/or
menu bars.

\paragraph*{Commands} \CLIM{} supports the separation of command
execution and command invocation. A \CLIM{} user can invoke commands
either via typing it into an interactor, clicking on a menu entry or
implicitly invoking a presentation-translator by clicking on a
presentation. Commands can also be invoked explicitly by the
programmer.

% added ``Redisplay'' below so that the paragraph header harmonizes
% with the jargon used in the paragraph.
\paragraph*{Dialogs and Incremental Update/Redisplay} Incremental Redisplay goes
a bit further than Output Recording. With Incremental Redisplay, an
output record can not only reproduce content that was written to a
stream, the \CLIM{} programmer can also attach the code that generated
the content to the content itself. Whenever necessary, the application
programmer can ask an output stream to update itself. \CLIM{} will
query all drawn elements for obsolescence, and if necessary, rerun the
code to produce fresh output.

\noindent\hrulefill

\noindent This is a large number of facilities to explore. The most
systematic way -- progressing from the lowest-level to the highest
-- would also be the lengthiest. Therefore, we start
with showing several facilities in action with the most fundamental
example in the realm of programming: Hello World.

\section{Our first application}

We will start with a few lines of code for the trivial Hello World example
to give the reader a test case to verify his \CLIM{} setup. It also
serves as a point of reference from where the reader can start his
explorations of more challenging \CLIM{} facilities. We do not try to
elaborate the \CLIM{} concepts in detail here, but simply use them
with a brief discussion. The confused reader may hope for a more
in-depth explanation in the following section. Please regard
\concept{pane}, \concept{application frame}, \concept{sheet},
\concept{sheet hierarchy}, \concept{graft} and \concept{top-level
  loop} as terms we will discuss later.

We provide extensive \CLIM{} specification references in
footnotes. The motivation for this is to show that all the relevant
information can be found in the \CLIM{} 2
specification\cite{clim-spec}. Before a good \CLIM{} programmer can
master any \CLIM{} concept, he has to get used to the style of writing
of the specification, as this is the most relevant work for
\CLIM{}. The best we can do in this short paper is provide pointers and
references and hope that the interested reader starts to explore the
surrounding text sections on his own.

After loading a \CLIM{} implementation, the package
\keyword{:clim-user} is available to absorb user code. This package is
a good start for experimentation and first steps. When proper
packaging is required, simply include the packages \keyword{:clim} and
\keyword{:clim-lisp} in your new package's \keyword{:use} list.

The central element of \CLIM{} application programming is the
\concept{application-frame}. An application frame is defined via
\code{define-application-frame}.\footnote{See Section 28.2 ``Defining
  and Creating Application Frames'' in \cite{clim-spec}.} Here is
the application frame definition for Hello World:
\lstset{style=inlinestyle}
\lstinputlisting{hello-world-def-app}

\begin{figure*}
\lstset{style=framestyle}
\lstinputlisting{hello-world-handle-repaint}
\caption{\method{handle-repaint} for \class{hello-world-pane}}\label{hello-world-repaint}
\end{figure*}
\code{define-application-frame}'s basic syntax is similar to \code{defclass} because
\code{define-application-frame} also generates classes. In this case,
it creates a frame class \class{hello-world} that has no superclass
except \class{frame} (which is added automatically).

With \code{:pane}, we define a \concept{top-level-pane} that becomes
the content of a fresh window that belongs to an application
frame.  Although the usual case is for an application frame to
correspond to a top level window, sometimes an application frame is swallowed by another
application and only space in another existing window is
reserved. For instance, a web site management tool might swallow a
text editor, so that the user has the option to edit web sites without
switching to another application.

% \footnote{The graft is the root of a sheet hierarchy and on most
%   windowing system, all children of a graft are mapped to windows.}
% With elaborating on details, this means that the content of the
% top-level-pane becomes the content of the application
% window.\footnote{This is not 100\% correct. A frame might decorate
%   the top-level-pane with a menu-bar or other refinements. The pane
%   adopted by the graft is then a composite pane containing a
%   menu-bar and the original top-level-pane.}

The list given after the \keyword{:pane} option is a form which is
evaluated when an instance of the \class{hello-world} class is
created. We use \method{make-pane} to construct a pane as the
top-level-pane for frame instances. \method{make-pane} is a
constructor for panes.\footnote{See Section 29.2 ``Basic Pane
  Construction'' in \cite{clim-spec}.} We can treat it as an analog to
\code{make-instance} especially made for pane classes.  Let us have a
look at the definition of \class{hello-world-pane}.

\lstset{style=inlinestyle} \lstinputlisting{hello-world-defclass} 

The one and only superclass of \class{hello-world-pane} is
\class{clim-stream-pane}.\footnote{See Section 29.4 ``CLIM Stream
  Panes'' in \cite{clim-spec}.} As there are no additional slots, an
experienced \CLOS{} programmer might guess that we will use
\class{hello-world-pane} solely for method specialization.  Before doing so,
let us have a look what we have actually
inherited from \class{clim-stream-pane}:\footnote{Internal classes
  removed from listing.}

\lstset{style=inlinestyle}
\begin{lstlisting}
CLIM-USER> (describe (find-class 
                      'clim-stream-pane))
DIRECT-SUPERCLASSES: 
PERMANENT-MEDIUM-SHEET-OUTPUT-MIXIN
STANDARD-REPAINTING-MIXIN
STANDARD-EXTENDED-INPUT-STREAM
STANDARD-EXTENDED-OUTPUT-STREAM
STANDARD-OUTPUT-RECORDING-STREAM
SHEET-MULTIPLE-CHILD-MIXIN
BASIC-PANE
\end{lstlisting}

% would it be appropriate to define the phrase ``protocol class''
% here?  I'm not sufficiently confident in my CLIM fu to provide a
% definition myself. [2006/03/14:rpg]

\class{basic-pane} is the foundation of all pane classes. It provides
reasonable defaults for all protocol methods and inherits from the
protocol class \class{pane}. In turn, \class{pane} inherits from
\class{basic-sheet}. Hence, all panes we construct via
\class{basic-pane} automatically adhere to the sheet protocol.

% The standard extended-input provides \class{clim-stream-pane} with a
% concept called \concept{command-inversion}. This concept allows the
% treatment of events as if they were coming from a stream with
% blocking reads.

% The permanent-sheet-output-mixin class delivers all methods for the
% \concept{output protocol} \footnote{See Section 8.3 ``Output
%   Protocol'' in \cite{clim-spec}}. This protocol used for delivering
% output to the user's screen. The special feature of this mixin is
% that the output medium is always present, it's permanent. That
% means, there is always a screen on which graphic operations can be
% carried out. In contrast, there are also output classes, that does
% not guarentee an output medium, but more on that later.

Our \class{hello-world-pane} needs some methods to be useful. With
\method{handle-repaint} in \figurename~\ref{hello-world-repaint}, we
participate in the \concept{repaint protocol}.\footnote{See Section
  8.4 ``Repaint Protocol'' in \cite{clim-spec}} This method has two
tasks: paint the pane with the pane's background and draw the greeting
of \code{*application-frame*} in the center of the pane. The
\code{hello-world} frame instance is automatically available in the
context of \method{handle-repaint} via the dynamically scoped variable
\code{*application-frame*} and so it is possible to use the accessor
\code{greeting} to obtain the instance's slot content.

Two functions are needed to see this code in action:
\code{make-application-frame} and \code{run-frame-top-level}. The
former is a constructor for instances of frame classes, the latter for
running the \concept{top-level loop} of the application frame. The
top-level loop is used for command reading, execution and displaying
results. But all we need to know for the moment is that it makes the
frame visible before entering the loop.
\lstset{style=inlinestyle}
\begin{lstlisting}
(run-frame-top-level
 (make-application-frame 'hello-world))
\end{lstlisting}
This completes the Hello World example. In the next section, we catch
up to the details we skipped in this example.

\section{Basic Facilities}

\subsection{Geometry}

% The footnote describing ``protocol'' below seems to give a critical
% insight into the style and functioning of CLIM.  It should certainly
% be promoted out of footnote and into body text.  I'm inclined to
% think it should be promoted to the introduction.  The notion of
% Protocol is alluded to there, but not clearly described.
% [2006/03/14:rpg]

To \CLIM{}, geometry means \concept{regions}. A region is either bound
or unbound and has a dimensionality of either zero, one or two. That
corresponds to a point, a path or an area respectively. Regions can be
compared (region predicate protocol\footnote{\CLIM{} relies heavily on
  \CLOS{}. In \CLOS{}, the term \concept{protocol} means a set of generic
  functions that together form a coherent interface. A protocol
  specification not only includes the syntactic details of the
  function names and the number of function arguments, but also the
  functions' purpose and the semantics of the return values (and/or side
  effects) must be given in a textual specification.}) and composed
(region composition protocol).

Every bounded region has a \concept{bounding rectangle}. It is the
smallest rectangle that contains every point in the region. The
bounding rectangle of every bounded region can be accessed via the
\concept{bounding rectangle protocol}.

\CLIM{} supports \concept{affine transformations} of regions. Such
transformations can move, stretch and rotate a region. A
transformation is affine when every straight line remains straight
after transformation. Transformations can be composed arbitrarily. The
programmer can attach transformations to mediums and panes. In layout
panes, \CLIM{} uses transformations to map the coordinates of child
panes to the coordinate system of their parents. 

% This was attached to the previous paragraph, but doesn't seem to
% have anything to do with its topic.  I'm inclined to think that this
% could use some further expansions (have we adequately explained what
% a drawing setting is?) [2006/03/14:rpg]
All drawing settings can be changed either permanently, or temporarily
in the context of the \macro{with-drawing-options} macro.


\subsection{The Windowing Substrate} 

\CLIM{} does not directly talk to the window system. Instead, \CLIM{}
is layered on top of a windowing substrate.\footnote{formerly known as Silica.}
This substrate is a middleware between \CLIM{} on one side and the
host window system on the other.  This middleware layer provides a
portable windowing model that insulates higher levels of \CLIM{} and
\CLIM{} programmers from the details of the host window system. From
the perspective of a \CLIM{} programmer, talking to the window system
is equal to talking to this abstraction layer. The implementation
details are hidden in \concept{backends}, like in \mcclim{} the CLX
backend, which hides X11 details, or the beagle backend, which hides
details for Mac OS X's toolkit. These backends will use the services
of a host window system to provide efficient windowing, input and
output facilities. Thanks to this middleware, \CLIM{} is portable
across different host windowing systems.

This framework allows uniform treatment of the following GUI building
blocks:
\begin{itemize}
\item Windows like those in X, Mac OS X and Microsoft Windows.
\item Gadgets typical of toolkit layers, such as Gtk+, QT or Mac OS X's
  toolkit. The backend provides a frame manager which takes care of
  mapping the abstract gadget types found in \CLIM{} to appropriate
  gadget with a native look and feel.
\item Structured graphics like output records and an application's
  presentation objects
\end{itemize}

The central abstraction specified by \CLIM{} is the \concept{sheet}. A
sheet is a surface that can be painted on and to which input gestures
can be directed, and that lives in a hierarchy of other such objects.
To get used to the notation of sheets, you can think of them as
swallowable windows.

The fundamental notion in \CLIM{} is the nesting of sheets within
another sheet called a windowing relationship. In a windowing
relationship, a parent sheet provides space to or groups a number of
other children sheets. The \concept{sheet protocols} specify
functionality for constructing, using, and managing hierarchies of
sheets.

Sheets have the following properties:
\begin{description}
\item[parent/children:] a sheet is part of a windowing hierarchy and
  maintains links to its parent and its children.
\item[coordinate system:] a sheet has its own coordinate system that
  might have a different scaling or orientation than its parent.
\item[transformation:] via a sheet transformation these differing
  sheet coordinate systems are mapped into coordinate system of their
  parents.
\item[clipping region:] defines an area within a sheet's coordinate
  system that indicates the area of interest.
\end{description}

\paragraph*{Window hierarchies} Sheets participate in a number of
protocols. The windowing protocols describes the relationship between
sheets. Every sheet has a parent, a sheet might have one or more
children. A sheet can be adopted by a parent sheet and disowned later.
A sheet is grafted when it is connected to a \concept{graft} either
directly or through it's ancestors. A graft is a special kind of sheet
that stands in for a host window, typically a root window (i.e.{} screen
level). A sheet is attached to a particular host window system by
making it a child of an associated graft. A host window will be
allocated for that sheet; the sheet will then appear to be a child of
the window associated with the graft.

Sheet hierarchies are displayed and manipulated on particular host
window systems by establishing a connection to that window system and
attaching them to an appropriate place in that window system's window
hierarchy. Ports and grafts provide the functionality for managing
this process. A port is a connection to a display service that is
responsible for managing host window system resources and for
processing input events received from the host window system.

\paragraph*{Window input} The input architecture of the windowing
substrate allows sheets to receive any window event from the host
windowing system. The event class hierarchy descends from the class
\code{event} to \code{device-event} (including all keyboard and mouse
actions), \code{window-event} (window-size-changed and window-repaint
events), \code{window-manager-events} (window-deleted) to artificial
\code{timer-events}. See 8.2 ``Standard Device Events'' in
\cite{clim-spec}.

The function pair \code{dispatch-event} and \code{handle-event} is the
center of all event handling. \code{dispatch-event} is intended to be
called when an event is to be dispatched to a client either
immediately or queued for later processing in an event queue. On the
other side, \code{handle-event} is intended for further
specialization, so the application developer can implement special
policies for selected events. For instance, when a sheet notices
through a \code{window-configuration-event} that the sheet's size
has changed, it might redo its layout for its children.

% There are two mixins that specialize on the
% \code{window-repaint-event} class as event argument to
% dispatch-event. The mixins are named standard-repaint-mixin and
% immediate-repainting-mixin. The first queues the repaint events when
% they arive via dispatch-event and redirect calls to handle-event
% with repaint events to the method handle-repaint. The latter mixin
% also redirects handle-event to handle-repaint for repaint events,
% but in contrast to the former it also immediately calls
% handle-repaint for any events ariving via dispatch-event. See Figure
% \ref{fig1}.

% \paragraph*{The repaint protocol} By demonstrating the use of
% handle-event with the repaint mixins, we have actually shown smaller
% protocol at work: the \concept{repaint protocol}. It's usually
% triggered by events generated from the host window system via the
% handle-event mechanism, but the \CLIM{} programmer can also trigger
% a repaint manually via the \code{repaint-sheet} function. The
% programmer should specialize on handle-repaint for his own classes.
% We have done that in the Hello World example.

\paragraph*{Window output}

All drawing operation happen on a \concept{medium}. This object
captures the state of the drawing like foreground, background, line
style, and the transformation which is applied to the coordinates
before drawing. Every medium is associated with a sheet. In turn, a
sheet must be associated with a medium whenever drawing operation
should be executed. Many \CLIM{} sheets are associated with a medium
permanently. \code{sheet-medium} obtains the medium associated with a
sheet.
% \footnote{Notice that the \CLIM{} specification \cite{clim-spec}
%   specifies draw-text, draw-line, draw-rectangle, etc. as functions
%   that operate on mediums. \mcclim{} names these functions
%   medium-draw-text, medium-draw-line, medium-draw-rectangle, and so
%   on. The functions that are available in \mcclim{} without the
%   ``medium-'' prefix operate on sheets. These are trampoline
%   functions that call the appropriate medium-prefixed function on
%   the sheet's medium. According to the \mcclim{} authors, this is done
%   for optimization.}

% in the topic sentence here, should it read ``of sheets'' or ``of
% mediums'' (media?).  I'm not sure, but if ``sheets'' is meant, we
% should probably have some transition wording here to explain how we
% got from discussing mediums above to discussing sheets here.
% [2006/03/14:rpg]

The graphic output capabilities of sheets range from simple line style
and text style customization over rendering various geometrical
shapes, a color model capable of doing alpha blending, composable
affine transformations to pattern, stencil and tiling filling, and
pixmaps usage. The features of the \concept{output protocol} are
specified briefly in Section 8.3 ``Output Protocol''and more precisely
in Chapters 10-14 of \cite{clim-spec}.

\CLIM{} lives in an idealized world in terms of graphics operations. A
\CLIM{} programmer can use an infinitely long and wide drawing pane
with arbitrarily precise resolution and continuously variable 
opacity. As rendering devices with these properties are rare,
we need to render the idealized graphic description to a device
with finite size and a fixed drawing precision. The rendering rules
are specified in Section 12.4 ``Rendering Conventions for Geometric
Shapes''of \cite{clim-spec}.

\section{Building Applications} 

In this section, we explain a number of the necessary ingredients for
building applications in \CLIM{}. We will illustrate frames, panes,
and simple commands with two examples: a color editor and a simple
line and text drawing program.

\subsection{Application Frames} 

Frames are the central abstraction defined by the \CLIM{} interface
for presenting an application's user interface. Many of the high level
features and facilities for application building provided by \CLIM{}
can be conveniently accessed through the frame facility.

Frames are typically displayed as top level windows on a desktop.
\concept{Frame managers} provide the machinery for realizing frames on
particular host window systems. A frame manager acts as an mediator
between the frame and what is typically called a desktop manager, or
in X terminology, a window manager. The \concept{frame manager} is
responsible for attaching the pane hierarchy of a frame to an
appropriate place in a sheet hierarchy (and therefore to a host window
system window hierarchy) when the frame is adopted.

To build a user interface, an application programmer defines one or
more frame classes. These frame classes define a number of frame
properties including application specific state and a hierarchy of
panes (i.e.{} user interface gadgets and regions, for interacting with
the users). Frame classes also provide hooks for customizing
application behavior during various portions of the frame protocol.
For example, an \keyword{:after} method on generic functions in the
frame protocol can allow applications to manage application resources
when the frame is made visible on some display server.

\CLIM{} is able to show dialog windows, but the code that brings them
up is usually quite different from the code that is used to generate
the content of application frames. This is unusual for a windowing
toolkit as most of them unify the generation of dialog content and
content of other window types.

\CLIM{} generates a dialog with the appropriate input gadget as
consequence of a series of input requests. Thanks to the stream
facility, the programmer can actually request input synchronously with
a blocking read request. He does not have to take care of
asynchronously handling confirmation or cancel button clicks.  For
instance, the programmer requests a string from the user and the user
is presented with a prompt, an editable text field, and two buttons
for confirmation and canceling. 
The string requesting function returns only after the user hits the
confirmation button.  The
programmer can directly use the function's return value which is the
string provided by the user. 
% Is the following rewrite correct?   Seems like in the actual code an
% abort-gesture is signaled, but it is handled by invoking an abort,
% if no abort-gesture handler is found.  [2006/03/14:rpg]
% OLD:
% Clicking the cancel button is dealt with by throwing to an
% \code{abort} tag.
Clicking the cancel button is dealt with by signaling an abort-gesture
condition.

From the caller's perspective, an attempt to separate application
frames and dialogs could be: a dialog window itself is side-effect
free with respect to the application state and therefore the whole
sense of calling a dialog creation routine must arise from the values
it returns. For example, the code that modifies the state of the text
editor in a text-replace operation does not rest with the callback
code of the Ok button in the dialog. This task rests with the code
that is executed after the dialog returns its values, namely the code
of the Search/Replace command.

% \footnote{This isn't a sharp distinction. We can also reinterprete
%   an application frame as something resulting from a command in
%   progress at a higher level. For instance, assume the application
%   frame to be the intermediate dialog resulting from the click on
%   your text editor icon of the desktop's launch menu/button, or to
%   be the result from typing the application name in your shell. A
%   more nifty example would be a text editor that allows embedded
%   graphics which can be edited in-place by graphics editor. The text
%   editor's application frame would then use another the graphic
%   editor's application frame intermediately returning the modified
%   graphic object.}

An intermediate dialog is something that is brought up to collect
additional information for (or before) an operation. When the user
selects the ``Search'' command, he is queried for a search string in
an additional dialog window; probably offering other search options
like case-insensitive search or backwards search. This is done in a
synchronous manner, blocking until the requested input is made
available by the user.

% The notation of sheets already capture the idea of windows.
% Remember, sheets are drawable entities that can receive user input.
% In fact, the entire screen area is also represented as sheet. The
% sheet representing the whole screen real-estate is named
% \concept{graft} in \CLIM{}\footnote{In X terminology, a graft would
%   be called root window.}. To create a new window, we have to order
% a graft to adopt a new sheet as its children. The new sheet is
% inserted into the sheet hierarchy below the graft (the sheet is said
% to be \concept{grafted}). But actually, we rarely do it that way.

% \CLIM{} provides a higher level abstraction of windows and what
% their purpose are. In \CLIM{}, we either have application frames or
% intermediate dialogs.

An application frame is an interface that provides the user with a
variety of commands to choose as his next step. For instance, the user
may choose from commands like Open, Save, Search, or Quit. The frame
is a long-living GUI object compared to dialogs, and there is no linear
execution path as there is in after a dialog as the user is free to
select any commands he likes as his next action.

Hence, the synchronous programming pattern for dialogs is more
convenient because after dialog confirmations there is a predetermined
path of execution, while an application frame has to be prepared to
handle an arbitrary sequence of commands.

% In contrast to dialogs, application frames tend to life longer than
% dialogs. Hence, application frames are usually required to maintain
% some state of some form. This is also the reason why
% \method{define-application-frame} was implemented on top of
% \macro{defclass}. A dialog does not need to maintain state as they
% are disposed too quickly.\footnote{They do have a state but this
%   state is not important to the programmer as it mostly consist of
%   state of the input gadgets, like the content of a text input
%   field. The amount of information contained in there is usually
%   neglectible, so we do not speak of state management here.}

% \paragraph*{Frame Managers} One important function provided by a
% frame manager is to adapt a frame to the look and feel of the host
% window manager. Before the top-level loop of a frame is entered, the
% frame is adopted by a frame manager provided by a \CLIM{} backend.


\subsection{Panes}

An application frame constructs its pane tree by specifying a
top-level pane in \macro{define-application-frame}. This pane is
usually a layout pane that contains more gadget and/or layout panes as
its children. With the help of layout panes, a pane hierarchy can be
constructed. The top-level pane (and the whole hierarchy when it is a
layout pane) is created when the application frame is adopted by a
frame manager and made visible to the user. The programmer can compose
an interface consisting of pre-defined gadget panes, layout panes, or
application-specific panes. \CLIM{} panes are rectangular sheets that
are analogous to the gadgets or widgets of other toolkits.

Panes and sheets as defined by the windowing substrate have in common
that they are associated with a region on screen, a parent, and
optional children. They differ in their usage of the input and output
capabilities. A sheet is passive and intended to be used by other,
active components, 
while a pane already contains this active part. 
For this reason,
panes are implemented as subclasses of \class{basic-sheet}
augmenting the class with an active part. For instance, a button-pane
actively draws its own button representation on its allotted screen
area and a click on the correct button area triggers a callback for
the button. A composite pane lays out its child elements and
requests them to draw themselves onto specific screen regions.

\CLIM{} comes with a set of predefined gadget panes. They consist of
push-button, toggle-button, slider, radio-box, text-field, text-editor
panes ready for use by the \CLIM{} application programmer. These
gadgets might be remapped to native system gadgets by the frame
manager, so a native look and feel is possible.\footnote{Only possible
  in \mcclim{} with the experimental beagle backend for Mac OS X.}

Each gadget pane class is associated with a set of generic functions
that act as callbacks do in traditional toolkits. For example, a pushbutton
has an ``activate'' callback method which is invoked when its button
is pressed. For this particular callback, a method named
\method{activate-callback} is invoked by default, and a \CLIM{}
programmer can provide a specialized method to implement
application-specific behavior for a subclassed button-pane. But except
in the case where the programmer needs a lot of buttons with related
behavior, creating a subclass for changing a single specific callback
is not economical. Hence upon gadget creation, the programmer can
specify an alternative callback method for any callback available. For
example, by
providing the \keyword{:activate-callback} initarg, the programmer can
change the callback to any regular or generic function. By convention,
any callback can be changed by providing an initarg keyword equal to
the callback's name. See Chapter 30 in \cite{clim-spec} for a listing
and description of available callbacks.

\CLIM{} also provides composite and layout panes. These pane types are
used for aggregating several child panes into a bigger single pane
that has a layout according to the requested directives.  For example,
\CLIM{} provides two pane classes, \class{hbox-pane} and
\class{vbox-pane}, that lay out their children in horizontal rows or
vertical columns respectively. The parent/child relations are managed
via the sheet's windowing protocol. If the user interface does not
change in ways unpredictable in advance (as in a user interface
builder for instance), the program does not have to do hierarchy
management via the windowing protocol. He is provided with a set of
convenience macros that allows elegant interfaces composed simply by
wrapping the respective pane construction code into the convenience
macros. % could we name the convenience macros here? [2006/03/14:rpg]

Application pane classes can be used for subclassing. They can be used
to present application specific data -- for instance by specializing
\method{handle-repaint} -- and to manage user interactions -- for
instance by specializing \method{handle-event}.

\subsection{Commands}

Most applications have a set of operations that can be invoked by the
user. In \CLIM{}, the command facility is used to define these
operations. Commands support the goal of separating an application's
user interface from its underlying functionality.  In particular,
commands separate the notion of an operation from the details of how
the operation is invoked by the user.

Application programmers define a command for each operation that they
choose to export as an explicit user entry point. A command is defined
to have a name and a set of zero or more operands, or arguments. These
commands can then be invoked using a variety of interaction
techniques. For example, commands can be invoked from menus, keyboard
accelerators, direct typein, mouse clicks on application data, or
gadgets.

The commands are processed in a REPL-like loop. Every application
frame has its own running top-level loop specified via
\keyword{:top-level} in \method{define-application-frame}. For a
\CLIM{} application, it is rarely necessary to change the default top
level loop.

The top-level loop becomes visible when an interactor-pane is added to
the user interface. Here the \CLIM{} user gains direct access to the
command loop. The loop steps are similar to read-eval-print:

\begin{enumerate}
\item Read a command.
\item Execute the command.
\item Run the \concept{display function} for each pane in the frame
associated with the top-level loop as necessary.
\end{enumerate}

Whenever a command is invoked other than by typing, an appropriate
command-invoking text appears after the command prompt
nonetheless. Here, the user can directly see how his commands are
synthesized from other invocation methods like pointer clicks or menu
item selections.

\section{Simple applications}

\subsection{Color Editor} In this next example, we define a frame for
color selection by manipulating their red, green, and blue components
separately. This example illustrates the use of gadget panes provided
by \CLIM{}. In the code in Figure \ref{fig-color-editor}, we define an
application frame using \method{define-application-frame}. As said
before, the syntax of this macro is similar to that of \class{defclass}. It
defines a new frame class which automatically inherits from the class
\class{frame} which provides most of the functionality for handling
frames.

One requirement must be fulfilled by all frame definitions: code to
generate a pane hierarchy must be supplied. The \keyword{:pane} option
is the simplest way to supply this code.  The \class{hello-world} frame
constructs a hierarchy with only one application pane via
\method{make-pane}. The pane hierarchy of color editor is more
interesting.

Every pane can request space via its initarg. A request for space
allocation is specified as a preferred size (\keyword{:height},
\keyword{:width}), a maximum size (\keyword{:max-height},
\keyword{:max-width}), and a minimum size (\keyword{:min-width},
\keyword{:max-width}). The special constant \constant{+fill+} can be
used to indicate the tolerance for any width or height.

The color editor frame uses three application slots for storing the
current RGB values as well as two panes for showing the currently
selected colors. The variable \variable{*application-frame*} is
dynamically scoped and is defined in any event handler as well as for
all code that is evaluated in the context of the \keyword{:pane(s)}
frame option.

\begin{figure*} \lstset{style=framestyle}
\lstinputlisting[firstline=7]{color-editor.lisp}
\caption{Color Editor}\label{fig-color-editor}
\end{figure*}

The code provided in the \keyword{:pane} option in Figure
\ref{fig-color-editor} uses all three kinds of panes provided by
\CLIM{}. It uses two application-panes to display colors, two layout
panes (\class{vbox-pane} and \class{hbox-pane}) and three gadget panes
(\class{slider-pane}).

In contrast to \class{hello-world}, we do not need a specialized pane
class here. We can use an application-pane for displaying chosen
colors.  An application pane supports graphics operations and invokes
generic functions on the pane when input events are received. For the
color editor, we only need its ability to refresh its background
color.

The \macro{vertically} and \macro{horizontally} convenience macros
provide an interface to the \class{hbox-pane} and \class{vbox-pane}
classes. Most \CLIM{} layout panes provide similar convenience
macros. The vertical box pane arranges its children in a stack from
top to bottom in the order they are listed at creation in the
vertically form. This pane type also supports inter-element space and
``pieces of glue'' at arbitrary points in the children sequence. In
the color editor frame, the \constant{+fill+} ``glue'' is used to
absorb all extra space when too much vertical space is allocated to
the vertical box. \CLIM{} also provides a horizontal box which does
the same thing except in the horizontal direction.

In the \macro{horizontally} macro in
\figurename~\ref{fig-color-editor}, we do not supply forms to be
evaluated directly. Instead, \method{horizontally} processes a form
wrapped up in a list. The first list element is a rational number
which denotes the amount of space the pane generated by the following
form is allowed to occupy in the resulting horizontal layout. In our
code in \figurename~\ref{fig-color-editor}, \class{hbox-pane}
generated from the \macro{horizontally} macro has a space requirement
on its own. The whole composite pane is forced to have a size 200
pixels high.

Now we are ready to turn to the gadget panes. The color editor uses
three sliders one for each component of RGB.
\method{make-color-slider} creates all three sliders that differ by
\variable{id}, \variable{initval} and \variable{label}. The first two
variables are handed to \method{make-pane} to construct the slider
panes. The remaining initargs for \method{make-pane} are shared across
all three entities. To decorate each slider-pane with a proper label,
each of them is wrapped up in a \class{label-pane} via the
\macro{labelling} convenience macro.

The slider gadget protocol defines two callback functions:
\method{drag-callback} is repeatedly invoked while the slider is being
dragged by the user, and \method{value-change-callback} is invoked
when the slider is released in a new location. Notice that this
specification is sufficiently abstract to allow a variety of different
look and feels for a slider. For example, no guarantee is made as to
whether the mouse button is held down during dragging, or whether the
mouse button is pressed once to start and again to stop dragging.

We use the gadget ID to distinguish between the red, green and blue
slider in the callback code. We could use three different callback
functions here, but such callbacks would have much more similarities
than differences, thus we do not do that here. Instead, we distinguish
the gadget by their \variable{id}.

The \method{drag-callback} method resets the background of
\variable{drag-feedback-pane} and delegates its redrawing to
\method{redisplay-frame-pane}, while \method{value-change-callback}
does the same for the frame's current color pane. These methods use
the \variable{id} argument of the gadget to determine which color
component was changed.

The color editor frame uses the \keyword{:menu-bar} option to indicate
that a menu-bar should be added to the application frame. Frame
commands defined with \code{:menu t} are accessible from the
menu-bar. In this example, we define only one command named
\code{com-quit} which is presented as ``Quit'' to the
user. \code{com-quit} is defined via
\macro{define-color-editor-command}. This macro is generated
automatically along \macro{define-application-frame} from the
application frame's name. \code{com-quit} simply closes the
application frame causing the termination of the frame's
top-level-loop. In the next example, commands will be explored in
greater detail.

\begin{figure*}[t] \lstset{style=framestyle}
\lstinputlisting{draw-frame-def-app}
\caption{define-application-frame for
\class{draw-frame}}\label{fig-draw-defapp}\label{fig-draw-handlerepaint}
\end{figure*}

We can invoke the color-editor with the regular
\method{run-frame-top-level}/\method{make-application-frame}
combination.  
\lstset{style=inlinestyle}
\begin{lstlisting}
(run-frame-top-level 
  (make-application-frame 'color-editor))
\end{lstlisting}


\subsection{A simple drawing application}

We move on to a simple application that draws
lines and inserts text interactively. Our simple drawing program
defines commands for various drawing operations and binds specific
input events to these commands.

The application frame is defined in
\figurename~\ref{fig-draw-defapp}. A different approach is used to
generate the pane hierarchy: instead of mixing the layout and pane
information, we use the \keyword{:panes} keyword to list all panes
that should be available in frame layouts. The \keyword{:layouts}
keyword combines them into several layouts. It is possible to define
multiple layouts with this option. We define two simple layouts,
\code{default-layout} and \code{alternative}.

% I was confused by the discussion of layouts here.  Looking at Figure
% 4, I do not see an alternative layout, and the default-default in
% the listing looks odd to me.  Is this the remainder of a discussion
% that was deleted or an extension yet to be written? [2006/03/19:rpg]

\begin{figure*}[t]
\lstset{style=framestyle}
\lstinputlisting{draw-frame-commands}
\caption{Commands for \class{draw-frame}}\label{fig-draw-commands}
\end{figure*}

\begin{figure*}
\lstset{style=framestyle}
\lstinputlisting{draw-frame-interfacing}
\caption{User Interfaces}\label{fig-draw-interfacing}
\end{figure*}

There are two options to \macro{define-application-frame} that we have
not seen before, \keyword{:command-definer} and \keyword{:top-level}
(both with there defaults). \keyword{:command-definer} is used to
specify a name for a command-defining macro for this frame
class. Passing \constant{t} to this option (its default) generates a
command definer named
\macro{define-\texttt{<}frame-name\texttt{>}-command}. We can use the
command-defining macro as a convenience macro for
\method{define-command} which is used to define frame commands, and 
will see it in action before long. \keyword{:top-level} specifies a
special form that is used as top level command loop. The top level is
responsible for dequeuing and executing commands that have been
invoked by the user. This loop is to a application frame what the REPL
is to a terminal.

This is the first example that does not use \class{clim-stream-pane}
(or one of its subclasses) as a pane class.\footnote{When using \mcclim{},
we have to do this as there are bugs in the behavior of
\class{clim-stream-pane} that have not been fixed yet.} Instead, we
compose our own drawing pane using
\class{standard-extended-input-stream}, \class{basic-pane} and
\class{permanent-medium-sheet-output-mixin}. The first class is used
to provide the application programmer with stream properties for the
draw-pane that are required for the convenience macro
\macro{tracking-pointer}. \class{basic-pane} is responsible for
handling all requests belonging to the sheet and pane protocols in a
standard manner. The last class is a mixin to tag the pane as
permanently visible on screen during its instances' lifetime. Most
static user interfaces use this mixin. Note that \mcclim{} is sensitive
to the order of the superclasses.\footnote{All stream classes like
\class{standard-extended-input-stream} must be listed before
\class{basic-pane}. Otherwise, no stream handling facilities
will be available.}

% The above footnote refers to what looks like a very big potential
% ``gotcha'' for the beginning McCLIM programmer.  Suggestion: promote
% this out of the footnote into body text as a main body paragraph
% here.  That paragraph should explain (1) why is it that putting
% basic-pane before standard-extended-input-stream will cause stream
% handling facilities to be crushed and (2) what exactly are the
% implications of not having stream handling facilities.  The phrase
% ``stream handling'' hasn't been defined, and it's not obvious what
% these facilities are. [2006/03/19:rpg]

For \class{draw-pane}, the \method{handle-repaint} method shown in
\figurename~\ref{fig-draw-handlerepaint} is straightforward. It
delegates background filling to the next less specific method and then
iterates through the lines and strings painting them. Here, we
implicitly defined the format of the \code{lines} slot and the
\code{strings} slot of the application frame class. The elements of
the list stored in \code{lines} are pairs of points, namely the start
and end point for a line. In \code{strings}, we store the text's
position as the \code{car} of a cons and the text as its \code{cdr}.

% \class{draw-frame} uses two options in
% \macro{define-application-frame} that we have not encountered
% before. The \keyword{:command-definer} option is used to specify a
% name for a command-defining macro for this frame class. Passing
% \constant{t} to this option, as in the example, indicates that a
% name of the form
% \macro{define-\texttt{<}frame-name\texttt{>}-command} should be
% used. The command-defining macro can then be used to define commands
% that are specialized to the defined frame class.


Figure \ref{fig-draw-commands} shows the command definitions. We have
commands for adding a text string, for adding a line and resetting the
draw pane. After every command, we update the drawing pane via the
auxiliary method \method{update-draw-pane}.\footnote{An experienced
\CLIM{} programmer would define a display-function for
\class{draw-pane}. This function is run after a command is
executed, and causes the display pane to be updated with any changes. We 
will save this technique for later examples.}

At this point, we can actually use the application; although not very
conveniently. The interactor pane can be used to invoke one of three
commands, either by typing the complete command including all its
parameters or by typing only the command name, then a dialog queries
the user for the missing command argument(s). Clicking on the menu bar
entries is another way to invoke commands.  When commands are invoked
through the menu bar, the user will be queried for the missing
argument(s) in the same way as if a command had been typed into the
interactor pane.
% also possible. Also in this case, the user is queried for
% the missing arguments.

But drawing by typing coordinates is not convenient. Therefore, we
attach these commands to other user interactions.  Figure
\ref{fig-draw-interfacing} defines input methods (methods for
\code{handle-event}) for pointer button
presses as well as key presses on the draw pane.  
Each handler invokes a tracking function (\code{track-line-drawing} and
\code{track-text-drawing}) that uses \macro{tracking-pointer} to
bypass the regular input distribution channels and to dispatch events
to user defined handlers.
% old text: [2006/03/19:rpg]
% Both handlers invoke
% the respective tracking function that uses \macro{tracking-pointer} to
% bypass the regular input distribution channels and to dispatch events
% to user defined handlers.

For \class{pointer-button-press-event}, which is used to draw lines, 
the input loop manages a
``rubber-banding'' line. %
% what is pointer-motion here?  Is it an event?  And it's not really
% the event that's invoked, to be absolutely clear.  The event is
% signaled, and the tracking-pointer macro provides a handler for that
% event.  So it is the code associated with :pointer-motion that is
% invoked...  I wasn't confident enough in my understanding of the
% precise organization of the (Mc)CLIM architecture to venture a
% rewrite, but there should be a clarification here. [2006/03/19:rpg]
The \keyword{:pointer-motion} is invoked
whenever the mouse pointer is moved by the user. The code attached to
\keyword{:pointer-motion} clears the previously-drawn line and draws a new
line with the new pointer position. It can easily undraw the old line
by the using the special ink \constant{+flipping-ink+}. When the user
confirms the line by releasing the pointer button, a command to the
application is synthesized via \method{execute-frame-command}
supplying all required parameters.

We provide a similar  input facility for text input. Whenever
the user hits a key in the draw-pane, the respective
\method{handle-event} calls \method{track-text-drawing} which attaches
the character entered to the mouse pointer.  As with the
rubber-banding line, the user can move the displayed string around while
he is free to append additional string characters by additional key
presses. He can confirm the text position with a mouse click causing a
command to be dispatched to the application frame that will add the text to
the application frame permanently.

As each of these methods invoke \code{execute-frame-command} passing
in a special command invocation form, this naturally leads to a
separation between the code that specifies how a command is invoked (menu-bar click, click on
draw-pane or typing in the interactor pane) and the code for command
execution (code bodies of \method{define-command}).

% \section{Shaping concepts}
% 
% \subsection{The Extended Input Stream vs. The Event-Based Input Protocol of Sheets}
% 
% \hrulefill
% 
% Usually, UI toolkits are event driven from design as most windowing
% systems use an event-driven design on their own. \CLIM{} supports a
% different style of programming. The difference command-inversion
% makes becomes mostly visible from the coding style of input
% processing (performing output is rarely an asynchronus operation).
% 
% The \CLIM{} programmer can treat the input as if they are coming
% from an stream. The extendend-input-stream even provides blocking
% read methods. The regular stream classes focus on character based
% interaction such as \method{stream-read-char} or
% \method{stream-read-line}, while the extended stream input class
% provides reading of mouse clicks and keyboard presses of the pane
% stream.
% 
% Later, when we talk about presentation-type, we will present a way
% to request objects with a special semantic from the user. For
% instance, we require the user to provide a file pathname, or
% clicking on a weekday item in a calendar. But more of
% command-inversion and presentations later.
% 
% \hrulefill
% 
% Even when you are familar with event-based GUI programming, this
% section is likely to be new to you. As said earlier, a \CLIM{}
% programmer can persue a different programming style with the
% extended input stream facilities. Any event-handler or command code
% can query an input stream for information.
% 
% \method{handle-event} from accept-demo shows an event handler for
% pointer release events. The handler calls two stream reading
% function \method{stream-read-line} and \method{stream-read-accept}.
% Notice that both functions are blocking.  The argument to
% \method{stream-read-accept} signals that the predefined presentation
% type 'pathname' is requested.
% 
% \lstinputlisting{accept-demo-handle-event} For the second pathname
% prompt, the user can either type a character-based representation of
% the pathname into the query window or he can select a pathname that
% is already present on screen with the pointer. accept-demo doesn't
% show any pathnames, so that is only a theoretical option, but we
% will see more about this option later in the presentation type demo.
% 
% The evaluation of both format forms is block by the evaluation of
% one of its argument, the stream-reading functions. Only after
% further interaction with the user, the accept call returns. In many
% other GUI toolkits, such a programming style is not possible as
% event handlers would not be allowed to block and therefore stall the
% event-handling thread.
% 

\section{High Level Facilities} 

In this section, we explain a number of higher level facilities
provided by \CLIM{}, including output recording, formatted output,
presentations, context sensitive input, and command
processors.\footnote{Many of these facilities are derived from work
  done at Symbolics on the Dynamic Windows (DW) project for
  Genera\cite{prog-ref-manual}. See \cite{presentation-manager} for
  more detailed information on the motivations and design details
  behind DW. Many of the original contributers to DW have participated
  in the redesign of these facilities for \CLIM{}.} We illustrate
these facilities in two examples: a directory lister and a simple
schedule browser.

\paragraph*{Output Recording} Many of the higher level facilities in
\CLIM{} are based on the concept of output recording. The \CLIM{}
output recording facility is simply a mechanism wherein a window
remembers all of the output that has been performed on it. This output
history (stored basically as a display list) can be used by \CLIM{}
for several purposes. For example, the output history can be used to
automatically support window contents refreshing (or ``damage
repaint'' events).  The application programmer has considerable
control over the output history. Output recording can be enabled or
suspended, and the history itself can be cleared or rearranged.

Output records can be nested, thereby forming their own hierarchy. The
leaves of this tree are typically records that represent a piece of
output, say the result of a call to \method{draw-rectangle} or
\method{write-string}. The intermediate nodes typically provide
additional semantics to the tree, such as marking a subtree of nodes
as resultant output of one particular phase of an application, for
instance row and column formatting information. Chapter 16 in
\cite{clim-spec} has all the details.

\paragraph*{Output Formatting} \CLIM{} provides a convenient table and
graph formatting facility, which is built on top of the output
recording facility. The key to these formatting tools (as opposed to,
say, \code{format}'s \code{~T} directive) is that they dynamically compute the
formatting parameters based on the actual size of the
application-generated output.

The application programmer uses these tools by wrapping any piece of
output-producing code with advisory macros that help the system
determine the structure of the output.

For example, start with a simple output function that shows some
information about the packages in the Lisp environment:
\lstset{style=inlinestyle}
\begin{lstlisting}
(defun show-package-info (stream)
 (dolist (package (list-all-packages)) 
   (write-string (package-name package) 
     stream)
   (write-string "  " stream) 
   (format stream "~D" 
           (count-package-symbols 
            package)) 
   (terpri stream)))
\end{lstlisting}
Any attempt to fix this function to produce tabular output by building
in a certain fixed spacing between the package name and symbol count
will either get caught by an unexpectedly long package name, or will
have to reserve way too much space for the typical case. In \CLIM{}, we
can use the code in Figure \ref{fig-formatting} to produce a neatly
formatted table for any set of package names.
\begin{figure*}
\lstset{style=framestyle}
\begin{lstlisting}
(defun show-packages (stream)
  (formatting-table (stream)
    (dolist (package (list-all-packages))
      (formatting-row (stream)
        ;; The first column contains the package name
        (formatting-cell (stream)
          (write-string (package-name package) stream))

        ;; The second color contains the symbol count, aligned with the right 
	;; edge of the column 
        (formatting-cell (stream :align-x ':right)
          (format stream "~D" (count-package-symbols-package)))))))
\end{lstlisting}
\caption{An output function that uses table formatting.}\label{fig-formatting}
\end{figure*}

% \paragraph*{Updating output} FIXME think of a GOOD example for
% updating output.

\paragraph*{Presentations} The next step up from preserving the mere
physical appearance of output done to a window is to preserve its
semantics. For example, when an application displays a Lisp pathname
on the screen via \code{(format t "\~{}A" path)}, the string
``/clim/demo/cad-demo.lisp'' may appear. To the user this string has
obvious semantic meaning; it is a pathname. However, to Lisp and the
underlying system it is just a text string. Fortunately, in many cases
the semantics can be recovered from the string. Thus the power of the
various textual cut-and-paste mechanisms supported by contemporary
computer systems. However, it is possible to improve upon the utility
of this lowest common denominator facility (i.e.{} squeezing everything
through its printed representation) by remembering the semantics of
the output as well as its appearance. This is the idea behind
presentations.

A presentation is a special kind of output record that maintains the
link between screen output and the Lisp data structure that it
represents. A presentation remembers three things: the displayed
output by capturing a subtree of output records, the Lisp object
associated with the output, and the presentation type of the output.
By maintaining this back pointer to the underlying Lisp data
structure, the presentation facility allows output to be reused at a
higher semantic level.

An application can produce semantically tagged output by calling the
\CLIM{} function \method{present}. For example, to display the
pathname referred to above as a presentation, the application would
execute:
\begin{lstlisting}
(present path 'pathname)
\end{lstlisting}
\method{present} captures the resulting output and the pathname object
in a presentation of type \code{'pathname}.

\paragraph*{Presentation Types} \CLIM{} defines a set of presentation
types, which are arranged in a super-type/subtype lattice like the CL
types. In fact, the presentation type hierarchy is an extension of the
CL type hierarchy. The reason that this extended type system is needed
is that the CL type system is insufficient from the UI perspective. For
example, the integer 72 might represent a heart rate in one
application and a Fahrenheit temperature in another, but it will
always be just an integer to Lisp.

The application programmer can define the UI entities of the
application by defining presentation types, thus extending the
presentation type library. By defining a presentation type, the
programmer can centralize all of the UI aspects of the new type in one
place, including output appearance and input syntax. As an example,
\CLIM{} defines a pathname presentation type that defines how a
pathname is displayed and how one is input. The pathname input side
provides pathname completion and display of possibilities. By
defining this behavior in one place and using it in all applications
that need to display or read pathnames, \CLIM{} helps build consistent
user interfaces.

Note that in the pathname output example given above \method{present}
invokes the standard pathname displayer defined by the presentation
type. However, since the presentation facility is simply based on the
output recording facility, presentation semantics can be given to any
output. The following example shows how the pathname object could be
associated with some graphics that were displayed on the screen.

\begin{lstlisting}
(with-output-as-presentation 
    (:object path
     :type 'pathname 
     :stream s) 
  (draw-rectangle* s 0 0 30 30))
\end{lstlisting}

\paragraph*{Context-Dependent Input} Once output is
semantically-tagged, it can be reused as semantically-meaningful
input. To achieve this, the application does not only have to tag its
output but it also has to provide additional semantic information when
doing input operations. For instance, whenever a pathname is required,
the program has to use a special method in contrast to reading a
string of characters.

The counterpart to \method{present} is \method{accept}. It is used to
establish an \concept{input context}.  The input context is a
presentation type that is appropriate for the current input point. For
example, if the application requires the user to input a pathname, it
can trigger an appropriate prompt, input parser and input context
with:
\begin{lstlisting}
(accept 'pathname :stream s)
\end{lstlisting}
Typically, this invokes the input reader (or parser) that was defined
for the pathname type and establishes an input context that indicates
that it is waiting for a pathname.

Once the input context is established, \CLIM{} automatically makes any
appropriate existing output available to the user via mouse gestures.
After calling \method{accept} as shown above, the user can move the
mouse over any presentation that is of type pathname (or is a subtype
of pathname), click on it, and the pathname object underlying the
presentation is returned as the value of the call to \code{accept}.

\paragraph*{Command Processors} \CLIM{} promotes the separation of
command execution code and command invocation code. The command
facility aids this task. Every application frame can define
application commands that are accessible via various input methods.
The most common are clicking on the command's entry in the menu bar or
a context menu, typing the command in the interactor pane, or the
application dispatches a command to itself (maybe triggered by other
ways of user input or network interactions, etc.). The latter case is
seen in \code{track-line-drawing} and \code{track-text-drawing} of the
\class{draw-frame} example. These methods generate ``Add Line'' and
``Add String'' commands as the result of event-handling on the
draw-pane.

A command has a name specified as string and a set arguments specified
as presentation types.  Looking back at the command ``Add String'' of
draw-frame, we see that this command takes a string and two integer as
arguments.  This type information is useful for partial command
parsing. Assume the user clicks on a menu entry or types only the
command name in the interactor. \CLIM{} notices that there are
arguments missing in the command and requests the missing ones in a
dialog via calls to \code{accept}. Mentioned early, \code{accept}
establishes an input context and so the user is able to fill the
missing argument with clicks on appropriate visible output objects.
Also keyboard users will find semantically-tagged input methods
convenient as implementations of presentation types can provide the
user with a completion facility.

In addition, the command processor is extensible by application
programmers. For example, the command processor can be extended to
support a ``noun then verb'' interaction style, where the user can
first click on a displayed presentation and then invoke a command that
is defined to take an argument of the selected presentation type.

\subsection{A Directory Browser}
\begin{figure*}
\lstset{style=framestyle}
\lstinputlisting{file-browser-all}
\caption{File Browser}\label{fig-file-browser}
\end{figure*}

The \class{dirlist-frame} application is a very simple file
system browser using presentation types.  It defines two panes, an
output pane to display directory contents and an input pane to handle
user typein.  For the output pane, the application defines a display
function. \code{dirlist-display-files} works in conjunction with the
top level command loop. By default, the command loop calls all
display functions after a command was processed to make all changes in
the application's data structures visible that the command execution
might have caused. When all display functions have updated their
panes, the command loop displays a prompt in the
interactor\footnote{Omitted when there is no interactor pane.}
and waits for the next command.

The \method{dirlist-display-files} display function iterates over the
contents of the current directory displaying the files one by one.
Each output line is produced by a call to \method{present}.
\method{present} creates the association between the text lines on the
screen and the Lisp pathname objects.

\class{draw-frame} has a single command ``Edit directory''. The
command's body interprets the pathname that it receives as a
directory, obtaining a list of the files contained therein. The
command simply updates the application variable
\variable{active-files} with the new list.

The \CLIM{} presentation substrate supports a general concept of
presentation type translation. This translation mechanism can be used
to map objects of one type into a different presentation type, if
appropriate. For example, it might be possible to satisfy an input
request for a pathname by selecting a computer user's login ID and
returning the pathname of the user's home directory. This would be
accomplished by defining a translator from a user-id presentation type
to the pathname type. The translator would consult the system's user
database to retrieve the home directory information to achieve its
conversion task.

The command loop of an application frame is regularly requesting
commands for processing and it uses \method{accept} for its request.
Hence, an input context for reading an object with the presentation
type \class{command} is established and all objects that can be used
as commands will become clickable on screen. Here, we can define a
presentation-to-command translator that translates a pathname into the
edit directory command.

The presentation-to-command translator in
\figurename~\ref{fig-file-browser} is very simple. It has the name
\class{pathname-to-edit-command} and converts the presentation type
\code{pathname} to the command \code{com-edit-directory} for the
command table of \class{file-browser}. The gesture option specifies
that it works with the select gesture, while the string supplied to
\keyword{:documentation} is used in the context menu.

This command translator has little work to do. The body of the
translator has to return a list of arguments that are handed to the
command \code{com-edit-directory}. But we do not need to any
conversion of the supplied object, as it is already a pathname. Thus,
the object is wrapped in a list and returned to the caller
which will apply \code{com-edit-directory} to the list.

A simple trick is used to dispatch an initial command to the
application frame. An after-method is provided for
\method{adopt-frame} which runs after the frame manager has adopted
the frame for displaying. This is different for an after-method on
\method{initialize-instance} as an after-method on
\method{adopt-frame} runs later, when the application frame instance
has already a command queue associated with it.

Since this application was defined with an interactor pane for user
input, the user can invoke the sole command by typing its name, ``Edit
Directory.'' Here, \CLIM{} supports for automatic command completion
becomes visible. Only the first letter has to be typed followed by the
complete action (usually Tab) and the command completion facility will
complete the input to ``Edit Directory''. At this point, the \CLIM{}
command loop will begin reading the arguments for that command, and
will automatically enter a pathname input context. Thus, the user can
fill in the required argument either by typing a pathname, or by
clicking on one of the pathnames visible in the display pane.

\subsection{Schedule Example}

In this example, we build a simple appointment browser. Since the user
of this application will frequently be dealing with the days of the
week, we start by defining a new presentation type weekday. This
simple presentation type, shown in Figure \ref{fig-scheduler1},
represents a day of the week as a number from 0 to 6. Each day number
is associated with an abbreviated day name, ``Mon,'' ``Tue,'' etc.

\begin{figure*}
\lstset{style=framestyle}
\lstinputlisting{scheduler-part1}
\caption{Scheduler: application frame, presentation type and commands}\label{fig-scheduler1}
\end{figure*}
\begin{figure*}
\lstset{style=framestyle}
\lstinputlisting{scheduler-part2}
\caption{Scheduler: display functions}\label{fig-scheduler2}
\end{figure*}

The weekday presentation type defines two basic pieces of behavior:
how a weekday is displayed, and how it is read as input. The macro
\macro{define-presentation-method} is intended for specializing
methods on presentation types.  Via this macro, we define a printer
and a parser function for the type \code{weekday}. As is the case with
most presentation types, the printer and parser are duals. That is,
the printer, when given an object to print, produces output that the
parser can interpret to arrive back at the original object.
\code{accept} does not need a parser in every case, and for simplicity
the programmer might choose not to provide a parser at all.

We took the easy way out with our parser. \CLIM{} provides a
completion facility and instead of reading the day name of the stream
and parsing it into an integer, we provided an exhaustive set of input
to object mappings. The \macro{completing-from-suggestions} macro
collects all suggestions made by the parser via
\code{suggest}. \code{suggest} takes an input and object as
suggestion. In our case, day name taken from \variable{*days*} is the
input and the number of the week \variable{i} is the object. \CLIM{}
will match the input given by the user with the suggestions and in
case of a match will return the corresponding object to the caller.

We define an application frame for the appointment browser in Figure
\ref{fig-scheduler1}. The frame defines state variables to hold the
list of appointments and the current day. This information is kept in
slots on the frame, so that multiple copies of the application can be
run, each with its own appointment list. The application represents
the appointment data as an alist containing an entry for each day of
the week, with each entry containing a list of the appointments for
the day. Test data is provided as a default initform.

Just as in the previous example, the appointment application defines
two panes, an interactor and an output display pane. The appointment
application defines two commands. The ``Show Summary'' command resets
the display back to the weekly summary mode by setting the
\variable{current-day} slot to \constant{nil}. The ``Select Day''
command sets \variable{current-day} to the value of an argument that
is specified to be a weekday. This presentation type specification
allows the command processor to make all presented weekday active when
it is filling in this argument, as well as, provide completion
assistance to the user.

The command ``Select Day'' has the following argument specification:
\begin{lstlisting}
((day 'weekday :gesture :select))
\end{lstlisting}
It takes an argument \code{day} of type \class{weekday}. Up until now,
our command arguments have looked similarly to specialized lambda
lists, but next to the type information the arguments of commands are
also allowed to have various keyword-value pairs for selection further
options. In this case, we supplied the value \code{:select} for the
option \keyword{:gesture}. The macro \macro{define-command} parses
this keyword-value pair and generate a presentation translator. More
precisely, a presentation-to-command translator is defined that is
equal in functionality to the one we have seen in the file-browser
example. Whenever a presentation of the type \code{pathname} is
selected (e.g.{} with pointer clicks) in a \class{command} input
context, it is translated into a command invocation of ``Select
Weekday''.

Finally, we turn to the display the appointment information. The
display function, \method{display-appointments}, shown in Figure
\ref{fig-scheduler2}, is somewhat more complex than our earlier
example. It can display two different sets of information: a weekly
summary showing the days of the week and the number of appointments
for each day, or a detailed description of one day's appointments.

\method{display-appointments} decides which set of information to
display by examining the application state variable
\variable{current-day}. The table formatting facility is used to
present the weekly summary information neatly organized. The daily
appointment list, by contrast, is displayed using
\method{write-string}. Note, however, that whenever a day of the week
is displayed, it is done with a call to \code{present} using the
weekday presentation type. This allows the printed weekdays to be
selected either as a command or as a weekday argument. This example
illustrates how an application with interesting UI behavior can be
constructed from a high-level specification of its functionality.

% \section{The benefits of the extended output stream}
% 
% At the first glance, the extended output stream facilities does not
% look more exciting than the output capabilities of sheets. In fact,
% the output streams use the output facility of sheets to do their
% job, thus the variaty of graphic operations is less or equal to that
% of the lower facility. But an output stream helps to organize output
% can do many things, we are not able to achieve that easy with the
% low-level output tools.
% 
% Output Recording, Table/Graph Formatting,and Incremental Redisplay
% are tools that are built on top of the stream characteristic in the
% extended output stream. We have already seen Table Formatting at
% work in the scheduler example.
% 
% \begin{figure*}
% \hrulefill
% \begin{lstlisting}
% (let (a-string 
%       sum-of-ints
%       (input-stream (frame-standard-input *application-frame*)))
%   (accepting-values (input-stream :own-window t)
%     (setq a-string (accept 'string :prompt "1st string" :stream input-stream))
%     (terpri input-stream)
%     (setq sum-of-ints 
%      (+ (prog1 (or (accept 'integer :prompt "1st integer" :stream input-stream) 0) 
%                (terpri input-stream))
%         (prog1 (or (accept 'integer :prompt "2nd integer" :stream input-stream) 0)
%                (terpri input-stream))))))
% \end{lstlisting}
% \hrulefill
% \caption{Hairy example of accepting-values demonstrating the use of the blocking \method{accept} method}\label{accepting-values-example}
% \end{figure*}
% 
% With the \macro{accepting-values} even more is possible.
% \figurename~\ref{accepting-values-example} demonstrates that it is
% possible to build up a whole dialog window simply by requesting a
% series of input types. It does that via \method{accepting-values}
% which brings up a seperate dialog window to request a string and two
% integers from the user.
% 
% The trick of \method{accepting-value} is dual-evaluation. Its body
% is evaluated twice. First to collect all accept calls and build up a
% dialog. The second evaluation happens after the user has entered all
% requested information and has hit the ``Ok'' button as confirmation.
% In the second evaluation the entered values are returned from the
% corresponding accept forms. Notice that we have to wrap
% \method{accept} in an or statement for \method{+}, because when the
% body is evaluated for the first time all \method{accept} forms
% return \constant{nil} causing an exception if handed to
% +.\footnote{In pratice, we would sum up the integers after
%   accepting-values, because this is much more easy to read later.}
% Please do not try to mimic the code shown in
% \figurename~\ref{accepting-values-example}. Nonetheless it is shown
% to give the reader an insight how this sudden violation of
% evaluation sequence is implemented.
% 
% We should emphasize that only the body of accepting-values is
% evaluated twice. The \method{let} statement in
% \figurename~\ref{accepting-values-example} does not return before
% \method{accepting-values} does. This is not a form of continuation.
% \method{accepting-values} effectively blocks the event thread just
% as a single \method{accept} would (\method{accepting-values} can
% therefore be seen as something that aggregates several accept
% statements). Despite this blocking, \CLIM{} manages to continue
% event handling.

\section{Conclusion}

The series of examples presented in this article illustrates the broad
range of functionality provided by \CLIM{}. The later examples,
especially, demonstrate that complex user interfaces can be built
economically and in a modular fashion using \CLIM{}. Many of the higher level
facilities make it possible to separate the issues involved in
designing an application's user interface from the functionality of
the application.

On the other hand, these higher level facilities are not
appropriate for all programmers. \CLIM{}'s lower level facilities and clean
modularization of the higher level facilities provide these programmers with
portable platform and a framework for implementing their own user
interface toolkits and frameworks. In addition, \CLIM{}'s use of \CLOS{}
to define explicit, documented protocols provides application
programmers with the opportunity to customize \CLIM{} and support
interfaces not anticipated by the \CLIM{} designers.

A free \CLIM{} implementation is available as \mcclim{}, found at
\url{http://common-lisp.net/project/mcclim/}. In addition to the
caveats in this document, note that \mcclim{} only works with an X
windows backend as of January 2006.  An experimental port called
beagle to the native toolkit of Mac OS X is in progress.

\section*{Acknowledgments}

\paragraph*{Original article} \CLIM{} represents the cooperative
effort of individuals at several companies. These individuals include
Jim Veitch, John Irwin, and Chris Richardson of Franz; Richard Lamson,
David Linden, and Mark Son-Bell of ILA; Paul Wieneke and Zack Smith of
Lucid; Scott McKay, John Aspinall, Dave Moon and Charlie Hornig of
Symbolics; and Gregor Kizcales and John Seely Brown of Xerox PARC.
Mark Son-Bell and Jon L. White have help us improve this paper.

\paragraph*{2006 update} Clemens Fruhwirth thanks the developers for
\mcclim{} for producing a free \CLIM{} implementation, and especially
Robert Strandh for answering so many questions in conjunction with
\mcclim{}.

\bibliographystyle{alpha}
\bibliography{guided-tour}

\end{document}