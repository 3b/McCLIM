% -*- Mode: LaTeX; Package: CLIM-USER -*-

\chapter {Input Editing and Completion Facilities}
\label {input-editing}

CLIM provides number of facilities to assist in writing presentation type parser
functions, such as an interactive input editor and some ``completion''
facilities.


\section {The Input Editor}

An input editing stream ``encapsulates'' an interactive stream, that is, most
operations are handled by the encapsulated interactive stream, but some
operations are handled directly by the input editing stream itself.  (See
Appendix~\ref{encapsulating-streams} for a discussion of encapsulating streams.)

An input editing stream will have the following components:

\begin{itemize}
\item The encapsulated interactive stream.

\item A buffer with a fill pointer, which we shall refer to as $FP$.  The buffer
contains all of the user's input, and $FP$ is the length of that input.

\item An insertion pointer, which we shall refer to as $IP$.  The insertion
pointer is the point in the buffer at which the ``editing cursor'' is.

\item A scan pointer, which we shall refer to as $SP$.  The scan pointer is the
point in the buffer from which CLIM will get the next input gesture object (in
the sense of \cl{read-gesture}).

\item A ``rescan queued'' flag indicating that the programmer (or CLIM)
requested that a ``rescan'' operation should take place before the next gesture
is read from the user.

\item A ``rescan in progress'' flag that indicates that CLIM is rescanning the
user's input, rather than reading freshly supplied gestures from the user.
\end{itemize}

The input editing stream may also have other components to store internal state,
such as a slot to accumulate a numeric argument or remember the most recently
used presentation history, and so forth.  These other components are explicitly
left unspecified.

The high level description of the operation of the input editor is that it reads
either ``real'' gestures from the user (such as characters from the keyboard or
pointer button events) or input editing commands.  The input editing commands
can modify the state of the input buffer.  When such modifications take place,
it is necessary to ``rescan'' the input buffer, that is, reset the scan pointer
$SP$ to its original state and reparse the contents of the input editor buffer
before reading any other gestures from the user.  While this rescanning
operation is taking place, the ``rescan in progress'' flag is set to
\term{true}.  The relationship $SP \leq IP \leq FP$ always holds.

The overall control structure of the input editor is:

\begin{verbatim}
(catch 'rescan                  ;thrown to when a rescan is invoked
  (reset-scan-pointer stream)   ;sets STREAM-RESCANNING-P to T
  (loop
     (funcall continuation stream)))
\end{verbatim}

where \arg{stream} is the input editing stream and \arg{continuation} is the
code supplied by the programmer, and typically contains calls to such functions
as \cl{accept} and \cl{read-token} (which will eventually call
\cl{stream-read-gesture}).  When a rescan operation is invoked, it has the
effect of throwing to the \cl{rescan} tag in the example above.  The loop is
terminated when an activation gesture is seen, and at that point the values
produced by \arg{continuation} are returned as values from the input editor.

The important point is that functions such as \cl{accept}, \cl{read-gesture},
and \cl{unread-gesture} read (or restore) the next gesture object from the
buffer at the position pointed to by the scan pointer $SP$.  However, insertion
and input editing commands take place at the position pointed to by $IP$.  The
purpose of the rescanning operation is to eventually ensure that all the input
gestures issued by the user (typed characters, pointer button presses, and so
forth) have been read by CLIM.  During input editing, the input editor should
maintain some sort of visible cursor to remind the user of the position of $IP$.

The overall structure of \cl{stream-read-gesture} on an input editing stream is:

\begin{verbatim}
(progn
  (rescan-if-necessary stream)
  (loop
    ;; If SP is less than FP
    ;;   Then get the next gesture from the input editor buffer at SP
    ;;   and increment SP
    ;;   Else read the next gesture from the encapsulated stream
    ;;   and insert it into the buffer at IP
    ;; Set the "rescan in progress" flag to false
    ;; Call STREAM-PROCESS-GESTURE on the gesture
    ;;   If it was a "real" gesture
    ;;     Then exit with the gesture as the result
    ;;     Else it was an input editing command (which has already been
    ;;     processed), so continue looping
    ))
\end{verbatim}

When a new gesture object is inserted into the input editor buffer, it is
inserted at the insertion pointer $IP$.  If $IP = FP$, this is accomplished by a
\cl{vector-push-extend}-like operation on the input buffer and $FP$, and then
incrementing $IP$.  If $IP < FP$, CLIM must first ``make room'' for the new
gesture in the input buffer, then insert the gesture at $IP$, then increment
both $IP$ and $FP$.

When the user requests an input editor motion command, only the insertion
pointer $IP$ is affected.  Motion commands do not need to request a rescan
operation.

When the user requests an input editor deletion command, the sequence of gesture
objects at $IP$ are removed, and $IP$ and $FP$ must be modified to reflect the
new state of the input buffer.  Deletion commands (and other commands that
modify the input buffer) must arrange for a rescan to occur when they are done
modifying the buffer, either by calling \cl{queue-rescan} or
\cl{immediate-rescan}.

CLIM implementations are free to put special objects in the input editor buffer,
such as ``noise strings'' and ``accept results''.  A ``noise string'' is used to
represent some sort of in-line prompt and is never seen as input; the
\cl{prompt-for-accept} method may insert a noise string into the input buffer.
An ``accept result'' is an object in the input buffer that is used to represent
some object that was inserted into the input buffer (typically via a pointer
gesture) that has no readable representation (in the Lisp sense);
\cl{presentation-replace-input} may create accept results.  Noise strings are
skipped over by input editing commands, and accept results are treated as a
single gesture.


\Defpredicate {interactive-stream-p} {object}

Returns \term{true} if \arg{object} is an interactive stream, that is, a
bidrectional stream intended for user interactions.  Otherwise it returns
\term{false}.  This is exactly the same function as in X3J13 Common Lisp,
except that in CLIM it is a generic function.

The input editor need only be fully implemented for interactive streams.

\Defprotoclass {input-editing-stream}

The protocol class that corresponds to an input editing stream.
\IfYouWantClass {an} {input editing stream} {input-editing-stream}

\Defpredicate {input-editing-stream-p} {object}

Returns \term{true} if \arg{object} is an \term{input editing stream} (that is,
a stream of the sort created by a call to \cl{with-input-editing}), otherwise
returns \term{false}.

\Defclass {standard-input-editing-stream}

The instantiable class that implements CLIM's standard input editor.  This is
the class of stream created by calling \cl{with-input-editing}.

\Mutable


\Defmacro {with-input-editing} {(\optional stream 
                                 \key input-sensitizer initial-contents class)
                                \body body}

Establishes a context in which the user can edit the input typed in on the
interactive stream \arg{stream}.  \arg{body} is then executed in this context,
and the values returned by \arg{body} are returned as the values of
\cl{with-input-editing}.  \arg{body} may have zero or more declarations as its
first forms.

The \arg{stream} argument is not evaluated, and must be a symbol that is bound
to an input stream.  If \arg{stream} is \cl{t} (the default),
\cl{*standard-input*} is used.  If \arg{stream} is a stream that is not an
interactive stream, then \cl{with-input-editing} is equivalent to \cl{progn}.

\arg{input-sensitizer}, if supplied, is a function of two arguments, a stream
and a continuation function; the function has dynamic extent.  The continuation,
supplied by CLIM, is responsible for displaying output corresponding to the
user's input on the stream.  The \arg{input-sensitizer} function will typically call
\cl{with-output-as-presentation} in order to make the output produced by the
continuation sensitive.

If \arg{initial-contents} is supplied, it must be either a string or a list of
two elements, an object and a presentation type.  If it is a string, the string
will be inserted into the input buffer using \cl{replace-input}.  If it is a
list, the printed representation of the object will be inserted into the input
buffer using \cl{presentation-replace-input}.


\Defmacro {with-input-editor-typeout} {(\optional stream \key erase) \body body} 

Establishes a context inside of \cl{with-input-editing} in which output can be
done by \arg{body} to the input editing stream \arg{stream}.  If \arg{erase} is
\term{true}, the area underneath the typeout will be erased before the typeout
is done.  \cl{with-input-editor-typeout} should call \cl{fresh-line} before and
after evaluating the body.  \arg{body} may have zero or more declarations as its
first forms.

The \arg{stream} argument is not evaluated, and must be a symbol that is bound
to a stream.  If \arg{stream} is \cl{t} (the default), \cl{*standard-input*} is
used.  If \arg{stream} is a stream that is not an input editing stream, then
\cl{with-input-editor-typeout} is equivalent to calling \cl{fresh-line},
evaluating the body, and then calling \cl{fresh-line} again.

\Defgeneric {input-editor-format} {stream format-string \rest format-args}

This function is like \cl{format}, except that it is intended to be called on
input editing streams.  It arranges to insert ``noise strings'' in the input
editor's input buffer.  Programmers can use this to display in-line prompts in
\cl{accept} methods.

If \arg{stream} is a stream that is not an input editing stream, then
\cl{input-editor-format} is equivalent to \cl{format}.


\subsection {The Input Editing Stream Protocol}

Input editing streams obey both the extended input and extended output stream
protocols, and must support the generic functions that comprise those protocols.
For the most part, this will simply entail ``trampolining'' those operations to
the encapsulated interactive stream.  However, some generic functions as
\cl{stream-read-gesture} and \cl{stream-unread-gesture} will need methods that
observe the use of the input editor's scan pointer.

Input editing streams will typically also implement methods for
\cl{prompt-for-accept} (in order to provide in-line prompting that interacts
correctly with input editing) and \cl{stream-accept} (in order to cause
\cl{accept} to obey the scan pointer).

The following generic functions comprise the remainder of the input editing
protocol, and must be implemented for all classes that inherit from
\cl{input-editing-stream}.


\Defmethod {stream-input-buffer} {(stream \cl{input-editing-stream})}

Returns the input buffer (that is, the string being edited) associated with the
\term{input editing stream} \arg{stream}.  This must be an unspecialized vector
with a fill pointer.  The fill pointer of the vector points past the last gesture
object in the buffer.  During input editing, this buffer is side-effected.  The
consequences of modifying the input buffer by means other than the specified API
(such as \cl{replace-input}) are unspecified.

\Defgeneric {stream-insertion-pointer} {stream}

Returns an integer corresponding to the current input position in the
\term{input editing stream} \arg{stream}'s buffer, that is, the point in the
buffer at which the next user input gesture will be inserted.  The insertion
pointer will always be less than \cl{(fill-pointer (stream-input-buffer
\arg{stream}))}.  The insertion pointer can also be thought of as an editing
cursor.

\Defgeneric {(setf stream-insertion-pointer)} {pointer stream}

Changes the input position of the \term{input editing stream} \arg{stream} to
\arg{pointer}.  \arg{pointer} is an integer, and must be less than
\cl{(fill-pointer (stream-input-buffer \arg{stream}))}.

\Defgeneric {stream-scan-pointer} {stream}

Returns an integer corresponding to the current scan pointer in the \term{input
editing stream} \arg{stream}'s buffer, that is, the point in the buffer at which
calls to \cl{accept} have stopped parsing input.  The scan pointer will always
be less than or equal to \cl{(stream-insertion-pointer \arg{stream})}.

\Defgeneric {(setf stream-scan-pointer)} {pointer stream}

Changes the scan pointer of the \term{input editing stream} \arg{stream} to
\arg{pointer}.  \arg{pointer} is an integer, and must be less than or equal to
\cl{(stream-insertion-pointer \arg{stream})}.


\Defgeneric {stream-rescanning-p} {stream}

Returns the state of the \term{input editing stream} \arg{stream}'s ``rescan in
progress'' flag, which is \term{true} if \arg{stream} is performing a rescan
operation, otherwise it is \term{false}.  All extended input streams must
implement a method for this, but non-input editing streams will always returns
\term{false}.

\Defgeneric {reset-scan-pointer} {stream \optional (scan-pointer \cl{0})}

Sets the \term{input editing stream} \arg{stream}'s scan pointer to
\arg{scan-pointer}, and sets the state of \cl{stream-rescanning-p} to
\term{true}.

\Defgeneric {immediate-rescan} {stream}

Invokes a rescan operation immediately by ``throwing'' out to the most recent
invocation of \cl{with-input-editing}.

\Defgeneric {queue-rescan} {stream}

Indicates that a rescan operation on the \term{input editing stream}
\arg{stream} should take place after the next non-input editing gesture is read
by setting the ``rescan queued'' flag to \term{true}.

\Defgeneric {rescan-if-necessary} {stream \optional inhibit-activation} 

Invokes a rescan operation on the \term{input editing stream} \arg{stream} if
\cl{queue-rescan} was called on the same stream and no intervening rescan
operation has taken place.  Resets the state of the ``rescan queued'' flag to
\term{false}.

If \arg{inhibit-activation} is \term{false}, the input line will not be
activated even if there is an activation character in it.

\Defgeneric {erase-input-buffer} {stream \optional (start-position \cl{0})}

Erases the part of the display that corresponds to the input editor's buffer
starting at the position \arg{start-position}.

\Defgeneric {redraw-input-buffer} {stream \optional (start-position \cl{0})}

Displays the input editor's buffer starting at the position \arg{start-position}
on the interactive stream that is encapsulated by the \term{input editing
stream} \arg{stream}.


\Defgeneric {stream-process-gesture} {stream gesture type}

If \arg{gesture} is an input editing command, \cl{stream-process-gesture}
performs the input editing operation on the \term{input editing stream}
\arg{stream} and returns \cl{nil}.  Otherwise, it returns the two values
\arg{gesture} and \arg{type}.

\Defmethod {stream-read-gesture} {(stream \cl{standard-input-editing-stream}) \key}

Reads and returns a gesture from the user on the \term{input editing stream}
\arg{stream}.

The \cl{stream-read-gesture} method must call \cl{stream-process-gesture}, which
will either return a ``real'' gesture (such as a typed character, a pointer
gesture, or a timeout) or will return \cl{nil} (indicating that some sort of
input editing operation was performed).  \cl{stream-read-gesture} must only
return when a real gesture was been read; if an input editing operation was
performed, \cl{stream-read-gesture} will loop until a ``real'' gesture is
typed by the user.

\Defmethod {stream-unread-gesture} {(stream \cl{standard-input-editing-stream}) gesture}

Inserts the gesture \arg{gesture} back into the input editor's buffer,
maintaining the scan pointer.


\subsection {Suggestions for Input Editing Commands}

An implementation of the input editor should provide a set of generally useful
input editing commands.  The exact set of these commands is unspecified, and the
key bindings for these commands may vary from platform to platform.  The
following is a suggested minimum set of input editing commands and key bindings,
taken roughly from EMACS.

\begin{tabular}{|l|l|}
\hline
\multicolumn{1}{c}{}                         & \multicolumn{1}{c}{\sl Suggested}   \\ 
\multicolumn{1}{c}{\sl Input editor command} & \multicolumn{1}{c}{\sl key binding} \\
\hline
Forward character   & {\tt control-F} \\
Forward word        & {\tt meta-F}    \\
Backward character  & {\tt control-B} \\
Backward word       & {\tt meta-B}    \\
Beginning of line   & {\tt control-A} \\
End of line         & {\tt control-E} \\
Next line           & {\tt control-N} \\
Previous line       & {\tt control-P} \\
Beginning of buffer & {\tt meta-<}    \\
End of buffer       & {\tt meta-<}    \\
Delete next character     & {\tt control-D} \\
Delete next word          & {\tt meta-D}    \\
Delete previous character & {\tt Rubout}    \\
Delete previous word      & {\tt m-Rubout}  \\
Kill to end of line       & {\tt control-K} \\
Clear input buffer        & {\sl varies}    \\
Insert new line           & {\tt control-O} \\
Transpose adjacent characters  & {\tt control-T}  \\
Transpose adjacent words       & {\tt meta-T}     \\
Yank from kill ring            & {\tt control-Y}  \\
Yank from presentation history & {\tt control-meta-Y} \\
Yank next item                 & {\tt meta-Y}     \\
Scroll output history forward   & {\tt control-V} \\
Scroll output history backward  & {\tt meta-V}    \\
\hline
\end{tabular}

An implementation of the input may also support ``numeric arguments'' (such as
{\tt control-0}, {\tt control-1}, {\tt meta-0}, and so forth) that modify the
behavior of the input editing commands.  For instance, the motion and deletion
commands should be repeated as many times as specified by the numeric argument.
Furthermore, the accumulated numeric argument should be passed to the command
processor in such a way that \cl{substitute-numeric-argument-marker} can be used
to insert the numeric argument into a command that was read via a keystroke
accelerator.

\Defun {add-input-editor-command} {gestures function}

Adds an input editing command that causes \arg{function} to be executed when the
specified gesture(s) are typed by the user.  \arg{gestures} is either a single
gesture name, or a list of gesture names.  When \arg{gestures} is a sequence of
gesture names, the function is executed only after all of the gestures are typed
in order with no intervening gestures.  (This is used to implement ``prefixed''
commands, such as the {\tt control-X control-F} command one might fix in EMACS.)


\section {Activation and Delimiter Gestures}

Activation gestures terminate an input ``sentence'', such as a command or
anything else being read by \cl{accept}.  When an activation gesture is entered
by the user, CLIM will cease reading input and ``execute'' the input that has
been entered.

Delimiter gestures terminate an input ``word'', such as a recursive call to
\cl{accept}.


\Defvar {*activation-gestures*}

The set of currently active activation gestures.  The global value of this must
be \cl{nil}.  The exact format of \cl{*activation-gestures*} is unspecified.
\cl{*activation-gestures*} and the elements in it may have dynamic extent.

\Defvar {*standard-activation-gestures*}

The default set of activation gestures.  The exact set of standard activation
is unspecified, but must include the gesture that corresponds to the
\verb+#\Newline+ character.

\Defmacro {with-activation-gestures} {(gestures \key override) \body body}

Specifies a list of gestures that terminate input during the execution of
\arg{body}.  \arg{body} may have zero or more declarations as its first forms.
\arg{gestures} must be either a single gesture name or a form that evaluates to
a list of gesture names.

If the boolean \arg{override} is \term{true}, then \arg{gestures} will override
the current activation gestures.  If it is \term{false} (the default), then
\arg{gestures} will be added to the existing set of activation gestures.
\cl{with-activation-gestures} must bind \cl{*activation-gestures*} to the new
set of activation gestures.

See also the \cl{:activation-gestures} and \cl{:additional-activation-gestures}
options to \cl{accept}.

\Defun {activation-gesture-p} {gesture}

Returns \term{true} if the gesture object \arg{gesture} is an activation
gesture, otherwise returns \term{false}.


\Defvar {*delimiter-gestures*}

The set of currently active delimiter gestures.  The global value of this must
be \cl{nil}.  The exact format of \cl{*delimiter-gestures*} is unspecified.
\cl{*delimiter-gestures*} and the elements in it may have dynamic extent.

\Defmacro {with-delimiter-gestures} {(gestures \key override) \body body}

Specifies a list of gestures that terminate an individual token, but not the
entire input, during the execution of \arg{body}.  \arg{body} may have zero or
more declarations as its first forms.  \arg{gestures} must be either a single
gesture name or a form that evaluates to a list of gesture names.

If the boolean \arg{override} is \term{true}, then \arg{gestures} will override
the current delimiter gestures.  If it is \term{false} (the default), then
\arg{gestures} will be added to the existing set of delimiter gestures.
\cl{with-delimiter-gestures} must bind \cl{*delimiter-gestures*} to the new set
of delimiter gestures.

See also the \cl{:delimiter-gestures} and \cl{:additional-delimiter-gestures}
options to \cl{accept}.

\Defun {delimiter-gesture-p} {gesture}

Returns \term{true} if the gesture object \arg{gesture} is a delimiter gesture,
otherwise returns \term{false}.


%% Use \tt instead of \cl, such the hairy \cl macro will blow chow
\section {Signalling Errors Inside {\tt present} Methods}

\Deferror {simple-parse-error}

The error that is signalled by \cl{simple-parse-error}.  This is a subclass of
\cl{parse-error}.  

This condition handles two initargs, \cl{:format-string} and
\cl{:format-arguments}, which are used to specify a control string and arguments
for a call to \cl{format}.

\Defun {simple-parse-error} {format-string \rest format-arguments}

Signals a \cl{simple-parse-error} error while parsing an input token.  Does not
return.  \arg{format-string} and \arg{format-args} are as for \cl{format}.

\Deferror {input-not-of-required-type}

The error that is signalled by \cl{input-not-of-required-type}.  This is a
subclass of \cl{parse-error}.

This condition handles two initargs, \cl{:string} and \cl{:type}, which specify
a string to be used in an error message and the expected presentation type.

\Defun {input-not-of-required-type} {object type}

Reports that input does not satisfy the specified type by signalling an
\cl{input-not-of-required-type} error.  \arg{object} is a parsed object or an
unparsed token (a string).  \arg{type} is a presentation type specifier.  Does
not return.
 

\section {Reading and Writing of Tokens}

\Defgeneric {replace-input} {stream new-input 
                             \key start end buffer-start rescan}

Replaces the part of the \term{input editing stream} \arg{stream}'s input buffer
that extends from \arg{buffer-start} to its scan pointer with the string
\arg{new-input}.  \arg{buffer-start} defaults to the current input position of
\arg{stream}.  \arg{start} and \arg{end} can be supplied to specify a
subsequence of \arg{new-input}; \arg{start} defaults to 0 and \arg{end} defaults
to the length of \arg{new-input}.

\cl{replace-input} must queue a rescan by calling \cl{queue-rescan} if the new
input does not match the old input, or \arg{rescan} is \term{true}.

The returned value is the position in the input buffer.

All input editing streams must implement a method for this function.

\Defgeneric {presentation-replace-input} {stream object type view
                                          \key buffer-start rescan 
                                               query-identifier for-context-type}

Like \cl{replace-input}, except that the new input to insert into the input
buffer is gotten by presenting \arg{object} with the presentation type
\arg{type} and view \arg{view}. \arg{buffer-start} and \arg{rescan} are as for
\cl{replace-input}, and \arg{query-identifier} and \arg{for-context-type} as as
for \cl{present}.

All input editing streams must implement a method for this function.  Typically,
this will be implemented by calling \cl{present-to-string} on \arg{object},
\arg{type}, \arg{view}, and \arg{for-context-type}, and then calling
\cl{replace-input} on the resulting string.

If the object does not have a readable representation (in the Lisp sense),
\cl{presentation-replace-input} may create an ``accept result'' to represent the
object, and insert that into the input buffer.  For the purposes of input
editing, ``accept results'' must be treated as a single input gesture.


\Defun {read-token} {stream \key input-wait-handler pointer-button-press-handler click-only}

Reads characters from the \term{interactive stream} \arg{stream} until it
encounters a delimiter or activation gesture, or a pointer gesture.  Returns the
accumulated string that was delimited by the delimiter or activation gesture,
leaving the delimiter unread.

If the first character of typed input is a quotation mark (\verb+#\"+), then
\cl{read-token} will ignore delimiter gestures until until another quotation
mark is seen.  When the closing quotation mark is seen, \cl{read-token} will
proceed as above.

If the boolean \arg{click-only} is \term{true}, then no keyboard input is
allowed.  In this case \cl{read-token} will simply ignore any typed characters.

\arg{input-wait-handler} and \arg{pointer-button-press-handler} are as for
\cl{stream-read-gesture}.

\Defun {write-token} {token stream \key acceptably}

\cl{write-token} is the opposite of \cl{read-token} given the string
\arg{token}, it writes it to the \term{interactive stream} \arg{stream}.  If
\arg{acceptably} is \term{true} and there are any characters in the token that
are delimiter gestures (see the macro \cl{with-delimiter-gestures}), then
\cl{write-token} will surround the token with quotation marks (\verb+#\"+).

Typically, \cl{present} methods will use \cl{write-token} instead of
\cl{write-string}.


\section {Completion}

CLIM provides a \concept{completion} facility that completes a string provided
by a user against some set of possible completions (which are themselves
strings).  Each completion is associated with some Lisp object.  CLIM
implementations are encouraged to provide ``chunkwise'' completion, that is, if
the user input consists of several tokens separated by ``partial delimiters'',
CLIM should complete each token separately against the set of possibilities.


\Defvar {*completion-gestures*}

A list of the gesture names that cause \cl{complete-input} to complete the
user's input as fully as possible.  The exact global contents of this list is
unspecified, but must include the \cl{:complete} gesture name.

\Defvar {*help-gestures*}

A list of the gesture names that cause \cl{accept} and \cl{complete-input} to
display a (possibly input context-sensitive) help message, and for some
presentation types a list of possibilities as well.  The exact global contents
of this list is unspecified, but must include the \cl{:help} gesture name.

\Defvar {*possibilities-gestures*}

A list of the gesture names that cause \cl{complete-input} to display a
(possibly input context-sensitive) help message and a list of possibilities.
The exact global contents of this list is unspecified, but must include the
\cl{:possibilities} gesture name.

\Defun {complete-input} {stream function 
                         \key partial-completers allow-any-input
                              possibility-printer (help-displays-possibilities t)}

Reads input from the user from the \term{input editing stream} \arg{stream},
completing over a set of possibilities.  \cl{complete-input} is only required to
work on input editing streams, but implementations may extend it to work on
interactive streams as well.

\arg{function} is a function of two arguments.  It is called to generate the
completion possibilities that match the user's input; it has dynamic extent.
Usually, programmers will pass either \cl{complete-from-possibilities} or
\cl{complete-from-generator} as the value of \arg{function}.  Its first argument
is a string containing the user's input ``so far''.  Its second argument is the
completion mode, one of the following:

\begin{itemize}
\item \cl{:complete-limited}---the function must complete the input up to the
next partial delimiter.  This is the mode used when the user types one of the
partial completers.

\item \cl{:complete-maximal}---the function must complete the input as much as
possible.  This is the mode used when the user issues a gesture that matches any
of the gesture names in \cl{*completion-gestures*}.

\item \cl{:complete}---the function must complete the input as much as possible,
except that if the user's input exactly matches one of the possibilities, even
if it is a left substring of another possibility, the shorter possibility is
returned as the result.  This is the mode used when the user issues a delimiter
or activation gesture that is not a partial completer.

\item \cl{:possibilities}---the function must return an alist of the possible
completions as its fifth value.  This is the mode used when the user a gesture
that matches any of the gesture names in \cl{*possibilities-gestures*} or
\cl{*help-gestures*} (if \arg{help-displays-possibilities} is \term{true}).
\end{itemize}

\arg{function} must return five values:

\begin{itemize}
\item \arg{string}---the completed input string.

\item \arg{success}---\term{true} if completion was successful, otherwise
\term{false}.

\item \arg{object}---the object corresponding to the completion, or \cl{nil} if
the completion was unsuccessful.

\item \arg{nmatches}---the number of possible completions of the input.

\item \arg{possibilities}---an alist of completions whose entries are a list
of a string and an object, returned only when the completion mode is
\cl{:possibilities}.  This list will be freshly created.
\end{itemize}

\cl{complete-input} returns three values: \arg{object}, \arg{success}, and
\arg{string}.  In addition, the printed representation of the completed input
will be inserted into the input buffer of \arg{stream} in place of the
user-supplied string by calling \cl{replace-input}.

\arg{partial-completers} is a list of characters that delimit portions of a name
that can be completed separately.  The default is an empty list.

If the boolean \arg{allow-any-input} is \term{true}, then \cl{complete-input}
will return as soon as the user issues an activation gesture, even if the input
is not any of the possibilities.  If the input is not one of the possibilities,
the three values returned by \cl{complete-input} will be \cl{nil}, \cl{t}, and
the string.  The default for \arg{allow-any-input} is \term{false}.

If \arg{possibility-printer} is supplied, it must be a function of three
arguments, a possibility, a presentation type, and a stream; it has dynamic
extent.  The function displays the possibility on the stream.  The possibility
will be a list of two elements, the first being a string and the second being
the object corresponding to the string.

If \arg{help-display-possibilities} is \term{true} (the default), then when the
user issues a help gesture (a gesture that matches one of the gesture names in
\cl{*help-gestures*}), CLIM will display all the matching possibilities. If it
is \term{false}, then CLIM will not display the possibilities unless the user
issues a possibility gesture (a gesture that matches one of the gesture names in
\cl{*possibilities-gestures*}).


\Defcondition {simple-completion-error}

The error that is signalled by \cl{complete-input} when no completion is found.
This is a subclass of \cl{simple-parse-error}.


\Defmacro {completing-from-suggestions} {(stream
                                          \key partial-completers allow-any-input
                                               possibility-printer
                                               (help-displays-possibilities \cl{t}))
                                         \body body}  

Reads input from the \term{input editing stream} \arg{stream}, completing over a
set of possibilities generated by calls to \cl{suggest} within \arg{body}.
\arg{body} may have zero or more declarations as its first forms.

\cl{completing-from-suggestions} returns three values, \arg{object},
\arg{success}, and \arg{string}

The \arg{stream} argument is not evaluated, and must be a symbol that is bound
to a stream.  If \arg{stream} is \cl{t} (the default), \cl{*standard-input*} is
used.

\arg{partial-completers}, \arg{allow-any-input}, and \arg{possibility-printer}
are as for \cl{complete-input}.

Implementations will probably use \cl{complete-from-generator} to implement
this.

\Defun {suggest} {completion object}

Specifies one possibility for \cl{completing-from-suggestions}.  \arg{completion}
is a string, the printed representation of \arg{object}.  \arg{object} is the
internal representation.

It is permitted for this function to have lexical scope, and be defined only
within the body of \cl{completing-from-suggestions}.


\Defun {complete-from-generator} {string function delimiters
                                  \key (action \cl{:complete}) predicate}

Given an input string \arg{string} and a list of delimiter characters
\arg{delimiters} that act as partial completion characters,
\cl{complete-from-generator} completes against the possibilities that are
generated by the function \arg{generator}.  \arg{generator} is a function of two
arguments, the string \arg{string} and another function that it calls in order
to process the possibility; it has dynamic extent.

\arg{action} will be one of \cl{:complete}, \cl{:complete-maximal},
\cl{:complete-limited}, or \cl{:possibilities}.  These are described under the
function \cl{complete-input}.

\arg{predicate} must be a function of one argument, an object.  If the predicate
returns \term{true}, the possibility corresponding to the object is processed,
otherwise it is not.  It has dynamic extent.

\cl{complete-from-generator} returns five values, the completed input string,
the success value (\term{true} if the completion was successful, otherwise
\term{false}), the object matching the completion (or \cl{nil} if unsuccessful),
the number of matches, and a list of possible completions if \arg{action} was
\cl{:possibilities}.

This function is one that will typically be passed as the second argument to
\cl{complete-input}.

\Defun {complete-from-possibilities} {string completions delimiters
                                      \key (action \cl{:complete}) predicate
                                           name-key value-key}


Given an input string \arg{string} and a list of delimiter characters
\arg{delimiters} that act as partial completion characters,
\cl{complete-from-possibilities} completes against the possibilities in the
sequence \arg{completions}.  The completion string is extracted from the
possibilities in completions by applying \arg{name-key}, which is a function of
one argument.  The object is extracted by applying \arg{value-key}, which is a
function of one argument.  \arg{name-key} defaults to \cl{first}, and
\arg{value-key} defaults to \cl{second}.

\arg{action} will be one of \cl{:complete}, \cl{:complete-maximal},
\cl{:complete-limited}, or \cl{:possibilities}.  These are described under the
function \cl{complete-input}.

\arg{predicate} must be a function of one argument, an object.  If the predicate
returns \term{true}, the possibility corresponding to the object is processed,
otherwise it is not.

\arg{predicate}, \arg{name-key}, and \arg{value-key} have dynamic extent.

\cl{complete-from-possibilities} returns five values, the completed input
string, the success value (\term{true} if the completion was successful,
otherwise \term{false}), the object matching the completion (or \cl{nil} if
unsuccessful), the number of matches, and a list of possible completions if
\arg{action} was \cl{:possibilities}.

This function is one that will typically be passed as the second argument to
\cl{complete-input}.


\Defmacro {with-accept-help} {options \body body}

Binds the dynamic environment to control the documentation produced by help and
possibilities gestures during user input in calls to \cl{accept} with the
dynamic scope of \arg{body}.  \arg{body} may have zero or more declarations as
its first forms.

\arg{options} is a list of option specifications.  Each specification is itself
a list of the form \arg{(help-option help-string)}.  \arg{help-option} is either
a symbol that is a \arg{help-type} or a list of the form \arg{(help-type
mode-flag)}.

\arg{help-type} must be one of:

\begin{itemize}
\item \cl{:top-level-help}---specifies that \arg{help-string} be used instead of
the default help documentation provided by \cl{accept}.

\item \cl{:subhelp}---specifies that \arg{help-string} be used in addition to
the default help documentation provided by \cl{accept}.
\end{itemize}

\arg{mode-flag} must be one of:

\begin{itemize}
\item \cl{:append}---specifies that the current help string be appended to any
previous help strings of the same help type. This is the default mode.

\item \cl{:override}---specifies that the current help string is the help for
this help type; no lower-level calls to \cl{with-accept-help} can override this.
(\cl{:override} works from the out-side in.)

\item \cl{:establish-unless-overridden}---specifies that the current help string
be the help for this help type unless a higher-level call to
\cl{with-accept-help} has already established a help string for this help type
in the \cl{:override} mode. This is what \cl{accept} uses to establish the
default help.
\end{itemize}

\arg{help-string} is a string or a function that returns a string.  If it is a
function, it receives three arguments, the stream, an action (either \cl{:help}
or \cl{:possibilities}) and the help string generated so far.

None of the arguments is evaluated. 
